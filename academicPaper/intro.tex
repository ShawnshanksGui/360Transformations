\section{Introduction}
\label{sec:introduction}

\subsection{Context and Motivations}

The popularity of navigable 360-degree video systems
has grown with the advent of both capturing systems
(including multi-camera with stitching systems) and interactive entertaining
systems (including head-mounted display devices).
However, the delivery of 360-degree video content, from the servers
to the end-users,
is still a challenge. 

\AD{Head-mounted devices (HMDs) display these videos in high resolution and frame rate, hence restricting the visible part to a viewer depending on the viewing angle, which is determined by the viewer's head direction. The viewing angle, measured horizontally or vertically from the head center, determines how wide or long the visible picture will be, which is specific for a given HMD. The observable part of the 360 degree video at a given time is referred to as a \textbf{Field of View (FoV) video}, which is characterized by the viewing angle (horizontal or vertical) and the observable picture's resolution. A \textbf{FoV center} is defined as the center point of the FoV video.}{}
%two videos (one per eye), each of them with a high frame rate and a high resolution (typically $1080\times 1200$ for state-of-the-art devices). 
%The combination of both videos
%form the \ac{FoV}, which is extracted
%from a wider spherical video with a resolution that is three to four times larger.

None of the current solutions for the delivery of 360-degree videos is entirely
satisfactory. Sending only \AD{FoV}{} videos is the least bandwidth-hungry implementation. However, it does
not enable fast
navigation within the 360-degree video: When the client moves the head, \AD{the FoV center changes, requiring a new FoV video to be immediately displayed. However, since a device has no knowledge about other parts of the spherical video, it has to notify the server about the changes and wait for reception of newly adjusted FoV video.}{}
%the device cannot immediately display any \ac{FoV} because it does not have information on the other
%parts of the full spherical video. The device has to notify the server and to wait
%for the reception of the newly adjusted \ac{FoV} videos. 
Another delivery implementation is to send the full spherical video
and to let the device
extract the \ac{FoV} videos. This solution enables fast navigation but the bandwidth requirements are
significant.

We study in this paper a solution where the server offers multiple \emph{versions} of the same
360-degree video. Each version is characterized by a \emph{\ac{QEC}}, which is a given
position in the spherical video. Around the \ac{QEC}, the quality of the video is maximum,
while it is lower for video areas that are far from the \ac{QEC}.
The client downloads the video version
such that
the \ac{QEC} of this video is close to the center of the \ac{FoV}.
The system is depicted in Figure~\ref{fig:deliverychain}.
This navigable 360-degree video delivery system has three advantages:
$(i)$ the bit-rate of the delivered video is lower than the original full-quality video.
$(ii)$ When the end-user does not move, the \ac{FoV} is extracted from the highest
quality spherical video.
And $(iii)$ when the end-user moves the head, the video does not stop. The device can
extract
a \ac{FoV} because it has the full spherical video. If the new \ac{FoV} is far from the
\ac{QEC} of the video version, the quality of the extracted video is lower but this
degradation is transient until the
device selects another video version with a closer \ac{QEC}.

\begin{figure}[h]
\centering
\begin{tikzpicture}


\tikzset{
     element/.style={
     	rounded corners,
     	rectangle,
  	 	thick,
  	 	draw=black,
  	 	minimum height=2cm,minimum width=2.5cm
     }
}

\tikzset{
	elementtitle/.style={
		rectangle,
		rounded corners,
		fill=gray!80,
		font=\footnotesize,
		text=white,
		anchor=north
	}
}

\tikzset{
	pics/equirec/.style n args={3}{
		code={
			\draw[fill=gray!30] (-0.0352778*#1, -0.019844*#1) rectangle (0.0352778*#1, 0.019844*#1);
			\draw[draw=none,fill=gray!70] (0.0088194*#2*#1-2*0.0088194*#1, 0.0066147*#3*#1 - 2*0.0066147*#1) rectangle (0.0088194*#2*#1 + 2*0.0088194*#1, 0.0066147*#3*#1 + 2*0.0066147*#1);
			\draw[draw,fill=none] (-0.0352778*#1, -0.019844*#1) rectangle (0.0352778*#1, 0.019844*#1);
			\draw[color=black,fill=black] (0.0088194*#2*#1, 0.0066147*#3*#1) circle (1pt);
		}		
	}
}

\def\convCmPt{0.0352778}
\def\convCmPtRec{0.019844}
\def\convCmPtRecThird{0.0066147}
\def\convCmPtFourth{0.0088194}

\tikzset{cross/.style={cross out, draw,
         minimum size=2*(#1-\pgflinewidth),
         inner sep=0pt, outer sep=0pt}}

\tikzset{
	fov/.pic ={
		\draw[densely dotted, thick, red!70!black] (-0.07,0.10) rectangle (0.37,-0.20);
%		\draw[fill=red] (0.2,-0.05) circle (2pt);
		\draw (0.15,-0.05) node[cross=2pt,red!70!black] {};
	}
}


\tikzset{
	vr/.pic = {
		\draw[rounded corners] (-0.0352778*#1, -0.019844*#1) rectangle (0.0352778*#1, 0.019844*#1);
		\draw[rounded corners, thick] (-0.032*#1, -0.019844*#1) rectangle (0.032*#1, 0.016*#1);
%		\draw(-0.019*#1,0) pic {fov};
%		\draw(0.019*#1,0) pic {fov};
		\node[font=\scriptsize,rectangle,red, draw=red, thick,
					densely dotted, anchor=east, inner sep=2pt,
					yshift=-1pt, xshift=-1pt] at (0,0) {L};
		\node[font=\scriptsize,rectangle,red, draw=red, thick,
					densely dotted, anchor=west, inner sep=2pt,
					yshift=-1pt,xshift=0.5pt] at (0,0) {R};
	}
}
		
\def\ecartElement{20pt}
\def\sizeSphere{11}
\def\ecartObjet{2}
\def\ecartYVersions{16pt}


% capturing system
\node[element] (0,0) (capturing) {};
\node[elementtitle, above=-5pt of capturing] {capturing};
\draw ([xshift=-\sizeSphere - \ecartObjet pt]capturing.east) pic {spherical=\sizeSphere};
\pgfdeclareimage[width=18 pt]{camera}{video-camera-icon-hi.png}
\node at ([xshift=21 pt]capturing.west) (camera1)
    {\pgfbox[right,center]{\pgfuseimage{camera}}};

% the arrow in the capture
\draw[-latex] (camera1.east) to ([xshift=-2*\sizeSphere - 2*\ecartObjet pt]capturing.east);


% ===== server
\node[element,right=\ecartElement of capturing] (server) {};
\node[elementtitle, above=-5pt of server] {\vphantom{pt}server};
\draw ([xshift=\sizeSphere + \ecartObjet pt]server.west) pic {spherical=\sizeSphere};
\draw ([xshift=-\sizeSphere - \ecartObjet pt]server.east) pic {equirec={\sizeSphere}{2}{1}};
\draw ([xshift=-\sizeSphere - \ecartObjet pt, yshift=\ecartYVersions]server.east) pic {equirec={\sizeSphere}{-2}{-1}};
\draw ([xshift=-\sizeSphere - \ecartObjet pt, yshift=-\ecartYVersions]server.east) pic {equirec={\sizeSphere}{0}{0}};

% the three arrows in the server
\draw[-latex] ([xshift=2*\sizeSphere + 2*\ecartObjet pt]server.west) to ([xshift=-2*\sizeSphere - 2*\ecartObjet pt]server.east);
\draw[-latex] ([xshift=2*\sizeSphere + 2*\ecartObjet pt]server.west) to ([xshift=-2*\sizeSphere - 2*\ecartObjet pt, yshift=\ecartYVersions]server.east);
\draw[-latex] ([xshift=2*\sizeSphere + 2*\ecartObjet pt]server.west) to ([xshift=-2*\sizeSphere - 2*\ecartObjet pt, yshift=-\ecartYVersions]server.east);

% between capture and server
\draw[-latex] ([xshift=2pt]capturing.east) to ([xshift=-2pt]server.west);


% ===== client
\node[element,right=\ecartElement of server] (client) {};
\node[elementtitle, above=-5pt of client] {\vphantom{pt}client};
% the equirecntagular
\draw([xshift=\sizeSphere + \ecartObjet pt, yshift=-\ecartYVersions]client.west) pic {equirec={\sizeSphere}{0}{0}};
% the old vr
%\draw ([xshift=-\sizeSphere - \ecartObjet pt]client.east) pic {vr=\sizeSphere};
% the fov
\pic[local bounding box=thisfov] at ([xshift=\sizeSphere + \ecartObjet pt, yshift=-\ecartYVersions]client.west) {fov};
%\node[anchor=west, font=\tiny,red, text width=23pt,align=center]
%		at ([yshift=10pt]client.west) (fovleg) {\ac{FoV} and \ac{FoV} center};
%
%\draw[red] (fovleg) -- (thisfov);

% vr headset
\pgfdeclareimage[width=24 pt]{vrheadset}{vr_icon.png}
\node at ([xshift=-28 pt]client.east) (headset)
    {\pgfbox[left,center]{\pgfuseimage{vrheadset}}};

% old vr
%

% the arrow in the client
\draw[-latex] 
%	([xshift=2*\sizeSphere + 2*\ecartObjet pt, yshift=-\ecartYVersions]client.west) to 
	(thisfov.60) to
		([xshift=-2*\sizeSphere - 2*\ecartObjet pt]client.east);


% between server and client
\draw[-latex] ([xshift=2pt, yshift=-\ecartYVersions]server.east) to ([xshift=-2pt, yshift=-\ecartYVersions]client.west);


\end{tikzpicture}
\caption{\ac{FoV}-adaptive 360-degree video delivery system: The server
offers video representations for three \acp{QEC}. The dark grey is the part of the video encoded at high quality and the light
gray the low quality. The \ac{FoV} video is the dotted red rectangle, and the \ac{FoV} center is the
cross}
\label{fig:deliverychain}
\end{figure}

%is characterized by an \emph{angle of vision}. It contains the full spherical
%video with an emphasis on the given angle of vision, \textit{i.e.},
%the part of the video that is in front of the angle of vision is at the highest quality and the video quality
%degrades for the other parts.

\subsection{Limitations of Previous Work}

The principle of delivering different qualities within a given navigable omnidirectional video is sketched
in a recent short paper by~\citet{ochi_live_2015}. Their proposal is based on the idea of video
tiling, which has been implemented for navigable panorama
video~\cite{sanchez_compressed_2015,wang_mixing_2014,gaddam_tiling_2015}:
The spherical video is mapped into an \emph{equirectangular} video, which
is cut into independent \emph{tiles} (typically $8\times 8$). The server offers several
video qualities for each tile. The client selects the quality of each tile according to
the \ac{FoV} of the end-user. This
solution has the same advantages as our proposal, but it neglects
the characteristics of 360-degree
videos. In an equirectangular video, the pixels
at the poles are over-sampled, which not only degrades the
performance of traditional video encoders~\cite{wojciechowski_h.264_2006,yu_framework_2015}, but also
makes equirectangular tiling less relevant. We consider instead geometric tiling on
other projections of spherical video.


The projection of spherical videos into a geometric layout generates
distortion on the resulting
full 2D video. Some layouts enable \ac{FoV} extraction without information
loss, typically on \emph{cube maps}~\cite{Ng2005} and
\emph{rhombic dodecahedron}~\cite{fu_rhombic_2009}. The previous work regarding
these
geometric layouts focuses on enabling efficient implementation of signal processing
functions~\cite{kazhdan_metric-aware_2010} and improving the video
encoding~\cite{tosic_low_2009}.
However, to the best of our knowledge, the
question of arranging various video qualities on the layout has not been studied so far.

Finally, a major content provider of 360-degree videos has recently released details about the
implementation of its delivery platform~\cite{facebook}. The depicted system is based
on the same idea as our proposal, where up to 30 versions of the same video are available depending on
\acp{QEC}. This implementation corroborates that, from an industrial perspective, the
extra-cost of
generating and storing multiple versions of the same video is worth the bandwidth
savings and the system usability. The authors projects the spherical videos into a \emph{pyramid}
where the
base is the front
view and the peak is in the back. Yet, this projection under-samples pixels near the peak,
which
results in
information loss and distorted extraction of \ac{FoV} videos.

\subsection{Our Contributions}

%
%of spherical videos into various geometrical layouts. We propose
%several ways to arrange the video qualities, which leverage the structure of the
%different layouts.
We study the navigable 360-degree video delivery system with a
focus on the arrangement of video qualities within the projections of
spherical videos into geometric layouts. Our contribution is twofold:
\begin{itemize}
\item We introduce a tool (released on open source in a public
repository\footnote{url is hidden for blind
preservation.}), which enables the projection from a spherical video
into the most studied  geometric layout (and vice versa). The tool has two main
features: it allows
any arrangement of video quality depending on the layout, and it
extracts the
\ac{FoV} video from any point in the sphere.
\item We provide a comprehensive comparison of the different geometric arrangements of
video quality on 360-degree video. We address the main questions raised
by~\citet{facebook}: which layout provides the best trade-off
between bit-rate and
quality, how many video versions should be offered by the server, and what geometric
quality arrangement
enables the smoothest navigation.
\end{itemize}

In this short paper, we restrict our study to essentials, and we provide only our main findings based on a
sample of the results. \textit{Here are our findings...}




%%% Local Variables:
%%% mode: latex
%%% TeX-master: "paper"
%%% End:
