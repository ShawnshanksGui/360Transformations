\begin{abstract}
The delivery and display of ultra high resolution 360-degree videos on Head-Mounted Displays (HMDs) presents a number of technical challenges. 360-degree videos are high resolution spherical videos that contain an omnidirectional view of the scene, however only a portion of this scene is displayed at a time on the user's HMD. The delivery of such videos wastes network resources since most of the pixels are never used. With high refresh rates, HMDs need to react in less than 10 ms to head movements. This prevents dynamic adjustments of video quality at the server based on the client's feedback. Instead, an entire 360-degree video scene needs to be delivered to the client to extract the appropriate fraction of this scene. To reduce a video bitrate, while still providing an immersive experience to users, a view-adaptive 360-degree video streaming system is proposed. In this system the server prepares multiple video representations that differ not only by bitrate, but also by qualities of different regions. The client chooses a representation for the next segment such that its bitrate fits the available throughput and full quality region matches the viewing direction. We investigate the impact of various spherical-to-plane projections and quality arrangements on the displayed video quality, showing that the cube map layout offers the best quality for the given bitrate budget. The evaluation with a dataset of users navigating in 360-degree videos demonstrates that segments needs to be short enough to enable frequent view switches.
\end{abstract}
%%% Local Variables:
%%% mode: latex
%%% TeX-master: "paper"
%%% End:
