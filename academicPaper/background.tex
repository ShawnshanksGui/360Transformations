% =============
\section{Background}

We now provide the background for our study.
First, we depict the overall architecture of the delivery system.
Then, we recall the main geometric layouts for spherical videos.

\subsection{Navigable 360-degree Video Delivery}

The principles of a navigable video delivery system are similar as in adaptive bit-rate
video systems such as \ac{DASH}. The server offers multiple versions of the same video
and the client
selects the most appropriate version according to some criteria. These versions
are cut into second-long segments such that the client can regularly switch from one
version to another. \AD{If the client moves the head before receiving a new video version,
he/she is displayed a new FoV video with gradually degrading quality that is adjusted to the head movement (as described in Section \ref{sec:context}). Exactly how many seconds should pass until receiving a new video version depends on the video content, the viewer's head movements (i.e., whether he/she watches a football match, news, or plays a computer game), and the experienced quality degradations variations, which needs to be investigated in the future work.}{}

At the client side, the end-user moves its head to decide the \ac{FoV}. The head movements
are \AD{forward and backwards, side to side, and should to shoulder, referred to as \emph{pitch}, \emph{yaw}, and \emph{roll}, respectively.}{}
 The center of the \ac{FoV} is a
point on the sphere, the size of the \ac{FoV} depends on the device (typically
around 100$^\circ$ in state-of-the-art devices), and the orientation of the extracted video
is related to the roll.

The complete analysis and evaluation of the navigable 360-degree video delivery system
is left for future work. Due to lack of space, we focus here on the geometric layout
of the video versions and the tile quality arrangement.


%\subsection{Geometric Layouts for Spherical Videos}

%The projection of a sphere into a plane (known as mapping) has been extensively studied
%for centuries. In this paper, we consider the four projections that are the most natural
%candidates for 360-degree video delivery. These layouts are depicted in Figure~\ref{fig:mapping}.

%\begin{figure}[ht]
%\centering
%\begin{tikzpicture}
%\def\sizeSphere{20}%pt
%\def\ecartY{-1.2}%cm
%\def\ecartX{6}

% da sphere
%\pic [local bounding box=spher]  at (0,0) {spherical=15};

% recantagular
%\pic [local bounding box=equi] at (-3,\ecartY) {equirectangular={\sizeSphere}{-1}{0}};

% cupe map
%\pic [local bounding box=cubemap] at (-1,\ecartY) {cubemap=\sizeSphere};

% pyramid
%\pic [local bounding box=pyra] at (1,\ecartY) {pyramid=\sizeSphere};

% rhombic
%\pgfdeclareimage[width=36 pt]{dodecahedron}{RhombicDodecahedron.png}
%\node at (3,\ecartY) (dodec)
%    {\pgfbox[center,center]{\pgfuseimage{dodecahedron}}};

%\def\unitused{0.22}

%\pic [local bounding box=dodeca] at (3,0.88*\ecartY) {dodecahedron=\unitused};

% links
%\draw[-latex] (spher.180) -| (equi);
%\draw[-latex] (spher.200) -| (cubemap);
%\draw[-latex] (spher.340) -| (pyra);
%\draw[-latex] (spher) -| (dodeca);

%\node[font=\scriptsize,anchor=north] at (equi.south) {equirectangular};
%\node[font=\scriptsize,anchor=north] at (cubemap.south) {cube map};
%\node[font=\scriptsize,anchor=north] at (pyra.south) {pyramid};
%\node[font=\scriptsize,anchor=north] at (dodeca.south) {\vphantom{y}dodecahedron};

%\end{tikzpicture}
%\caption{Projections into four geometric layouts}\label{fig:mapping}
%\end{figure}

%The advantages and shortcomings of every projection have been studied in the literature. From
%the images that are
%projected on an equirectangular, a cube map, and a rhombic dodecahedron, it is possible
%to generate a \ac{FoV}
%for any position and angle in the sphere. Indeed, no information is loss, because no pixel is under-sampled in
%this projection (no pair of pixels in the sphere is projected on these geometrical
%layouts in only one pixel). However,
%some pixels from the spherical image are over-sampled in the projected image. It is typically the case for
%the equirectangular layout, for which the projection generates a significant over-sampling at the poles. On the
%contrary, the pyramid is not a geometric layout for lossless projections. Some pixels (those who are in the back
%of the view face) are under-sampled, so two pixels can be projected into one pixel by interpolating their color
%values. A \ac{FoV} that is extracted for positions near the back can suffer from distortion. \AD{Hence, this is the least probable head orientation of the video viewer.}{} 