\section{Introduction}

The popularity of navigable 360-degree video systems has grown with the advent of both capturing systems (including multi-camera with stitching systems) and interactive entertaining systems (including head-mounted display devices). However, the delivery of 360-degree video content, from servers to the end-users, is still a challenge.

360-degree videos that are used in head mounted displays have large resolution and frame rate, thus resulting in very large file sizes. However, a viewer sees only a part of the video at a given time, based on the device viewing angle (i.e., Field of View (FoV)) that determines how long and wide the displayed picture is seen by the viewer. The viewing angle, measured horizontally or vertically from the head center, scales from 60-150 degrees for most of the currently available HMP devices. For a HMD device to display a FoV in full HD quality (1280x1080) to a viewer, it needs to deliver a 360-degree video in 3-6 times higher resolution (i.e., at least 4k at 3840x2160). Delivery of such videos with 35-68 Mbps bitrate to end users over the existing network infrastructure is quite challenging, requiring a significant increase of the Internet connection speed (from the currently used 5 Mbps for 720p content).

What do existing 360 video delivery systems do, are they solving this task? Is this extensively studied or is this field at the beginning?

In this paper we deliver the entire 360-degree video to the end user, projected into different geometrical layouts while varying the resolution of faces that correspond to potential viewing directions. The resolution of a face is set based on the distance of the viewing trajectory of the user' head from the center of the front face to the particular face direction )corresponding to the likelihood that the user turns his head to watch this part of the video).

Such a composed video has a reduced size compared to the same video ... (in equirectangular layout with tiles in different qualities, ... with all faces encoded in the same resolution.). Additionally, the quality of this video degrades with the distance from the viewer's previous head position, corresponding to the likelihood that the user turns his head to watch this part of the video.

What are the key questions that we intend to answer in this paper? How do our questions differ from the existing literature?

How do we answer these questions and bridge the gap in 360-degree video delivery? (What is the impact/scale of measurements/experiments/dataset compared to existing frameworks?)

What are our answers to these questions (observations)?

What are main contributions of our work?

Organization of the paper.

\section{Related work}

Review the most relevant work on the topic. What are the findings? Where is the missing gap?

Separate it in categories: tiling, spherical projections, and view-dependent streaming.

\section{Background}

\subsection{Navigable 360-degree Video Delivery}

Move the explanation and figure from Introduction here, and complement it with the existing text from this section.

\subsection{Geometrical layouts for Spherical Videos}

What is the motivation for projecting spherical videos into a plan and different geometrical layouts?

Explain each of the four projections with advantages and disadvantages, with layouts that corresponds to different viewing orientation.

\section{Our 360-degree video delivery tool}

This is what we propose ...
Implementation ...

\section{Results}

Evaluation details, results ...

\section{Discussion}

Discuss open questions and alternatives ....

\section{Conclusions}

\section{References}