\section{System architecture}


This section describes the system architecture of the proposed
navigable $360$-degree video delivery framework.

\parag{Server}The server takes as an input a $360$-degree video in
equirectangular format and transforms each frame into a desired
geometrical layout. Then, it creates $n$ different video
versions, each with a different \ac{QER} and encoded in $k$ different
bit-rates (see Figure~\ref{fig:newdelivery}). The server
splits all such encoded videos into segments, which are classified in
$n\!\times\!k$ representations (based on their respective bit-rate and
\ac{QER}), enabling clients to regularly switch from one
representation to another. The video quality around the
\ac{QEC} is the highest, while the remaining part is encoded at lower
quality.

\begin{figure}
   \centering
   \tikzsetnextfilename{new_delivery}
\begin{tikzpicture}

\tikzset{
     element/.style={
     	rounded corners,
     	rectangle,
  	 	thick,
  	 	draw=black,
  	 	minimum height=2cm,minimum width=2.5cm
     }
}

\tikzset{
	elementtitle/.style={
		rectangle,
		rounded corners,
		fill=titles,
		font=\footnotesize,
		text=white,
		anchor=north
	}
}

% this is the equirectangular with QEC and qualities
\tikzset{
	pics/equirec/.style n args={3}{
	% the first parameter is the size
	% the second parameter is the x coordinate of the QEC
	% the third parameter is the y coordinate of the QEC
		code={
			\draw[fill=midquality] (-0.0352778*#1, -0.019844*#1) rectangle (0.0352778*#1, 0.019844*#1);
			\draw[draw=none,fill=fullquality] (0.0088194*#2*#1-2*0.0088194*#1, 0.0066147*#3*#1 - 2*0.0066147*#1) rectangle (0.0088194*#2*#1 + 2*0.0088194*#1, 0.0066147*#3*#1 + 2*0.0066147*#1);
			\draw[draw,fill=none] (-0.0352778*#1, -0.019844*#1) rectangle (0.0352778*#1, 0.019844*#1);
			\draw[color=black,fill=black] (0.0088194*#2*#1, 0.0066147*#3*#1) circle (1pt);
		}
	}
}

\tikzset{
	emptyEquirec/.pic={
		\draw[fill=midquality] (-0.0352778*#1, -0.019844*#1) rectangle (0.0352778*#1, 0.019844*#1);
	}
}

\def\convCmPt{0.0352778}
\def\convCmPtRec{0.019844}
\def\convCmPtRecThird{0.0066147}
\def\convCmPtFourth{0.0088194}

\tikzset{cross/.style={cross out, draw,
         minimum size=2*(#1-\pgflinewidth),
         inner sep=0pt, outer sep=0pt}}

% here is for the FoV
\tikzset{
	pics/fov/.style n args={2}{
		% the first parameter is the size of the screen
		% the second parameter is the x coordinate of the FoV center (0, 1 or 2)
		code = {
			\draw (-0.5*#1, -0.28*#1) rectangle (0.5*#1, 0.28*#1);
			\draw[densely dotted, thick, red!70!black] (-0.45*#1 + #2*#1*0.24, 0.15*#1)
				 rectangle
			 	(-0.4*#1 + #2*#1*0.24 + 0.4*#1, -0.15*#1);
			\draw (-0.4*#1 + #2*#1*0.24 + 0.2*#1, 0.05*#1) node[cross=2pt,red!70!black] {};
	   }
	}
}


\tikzset{
	vr/.pic = {
		\draw[rounded corners] (-0.0352778*#1, -0.019844*#1) rectangle (0.0352778*#1, 0.019844*#1);
		\draw[rounded corners, thick] (-0.032*#1, -0.019844*#1) rectangle (0.032*#1, 0.016*#1);
		\node[font=\scriptsize,rectangle,red, draw=red, thick,
					densely dotted, anchor=east, inner sep=2pt,
					yshift=-1pt, xshift=-1pt] at (0,0) {L};
		\node[font=\scriptsize,rectangle,red, draw=red, thick,
					densely dotted, anchor=west, inner sep=2pt,
					yshift=-1pt,xshift=0.5pt] at (0,0) {R};
	}
}

% this is for three representations at different sizes for a given QEC
\tikzset{
	pics/threerep/.style n args={5}{
	% first parameter is a bool if bw must be written
	% second parameter is the QEC number (0 if not displayed)
	% third parameter is the segment number (0 if not displayed)
	% fourth parameter is the x pos of QEC
	% fifth paramerer is the y pos of QEC
		code={
			\pic[local bounding box=bigA] at (0,0) {equirec={\sizeBig}{#4}{#5}};
			\ifnum#1>0
		    	\node[font=\scriptsize, anchor=east, inner sep=0pt]
		    		at (bigA.west) (legbwhi) {high};
		    \fi
		    \ifnum#3>0
				\node[font=\scriptsize,anchor=south] at (bigA.north) {$s_{#3}$};
			\fi
%			\pic[local bounding box=medA] at ([yshift=\ecartMed]bigA.south) {equirec={\sizeMed}{#4}{#5}};
%			\ifnum#1>0
%				\node[font=\scriptsize,anchor=west, inner sep=0pt]
%					at (legbwhi.west |- medA) (legbwme) {\vphantom{g}med};
%			\fi
%			\pic[local bounding box=lowA] at ([yshift=\ecartLow]medA.south)
%				{equirec={\sizeLow}{#4}{#5}};
			\pic[local bounding box=lowA] at ([yshift=\ecartLow]bigA.south)
				{equirec={\sizeLow}{#4}{#5}};
			\ifnum#1>0
				\node[font=\scriptsize, inner sep=0pt, anchor=west]
					at (legbwhi.west |- lowA) {\vphantom{g}low};
			\fi
			\ifnum#2>0
				\node[font=\footnotesize,anchor=east] at (legbwhi.south west) {QEC$_#2$};
			\fi
		}
	}
}

% this is for the guy looking at the left
\tikzset{
	leftVR/.pic = {
	% the only parameter is the size
		% first the body
		\draw plot [smooth] coordinates {(0,0)
			(0.05*#1,0.2*#1)
			(0.25*#1,0.2*#1)
			(0.3*#1,0)};
		\draw (0,0) -- (0.3*#1,0);
		% then the head
		\draw[fill=white] (0.15*#1, 0.4*#1) ellipse (0.15*#1 cm and 0.2*#1 cm);
		% then the VR headset
		\draw[rounded corners=0.3*#1, fill=black] (-0.03*#1, 0.55*#1) rectangle
			(0.1*#1, 0.35*#1);
		\begin{scope}
			\clip (0.15*#1, 0.4*#1) ellipse (0.15*#1 cm and 0.2*#1 cm);
			\draw[fill=black] (0.1*#1, 0.5*#1) rectangle (0.3*#1, 0.46*#1);
			\draw plot [smooth] coordinates {(-0.01*#1,0.3*#1)
			 (0.04*#1,0.27*#1)
			 (0.09*#1,0.3*#1)};
		\end{scope}
	}
}

% this is for the guy looking at the right
\tikzset{
	rightVR/.pic = {
	% the only parameter is the size
		% first the body
		\draw plot [smooth] coordinates {(0,0)
			(0.05*#1,0.2*#1)
			(0.25*#1,0.2*#1)
			(0.3*#1,0)};
		\draw (0,0) -- (0.3*#1,0);
		% then the head
		\draw[fill=white] (0.15*#1, 0.4*#1) ellipse (0.15*#1 cm and 0.2*#1 cm);
		% then the VR headset
		\draw[rounded corners=0.3*#1, fill=black] (0.33*#1, 0.55*#1) rectangle
			(0.2*#1, 0.35*#1);
		\begin{scope}
			\clip (0.15*#1, 0.4*#1) ellipse (0.15*#1 cm and 0.2*#1 cm);
			\draw[fill=black] (0.2*#1, 0.5*#1) rectangle (0, 0.46*#1);
			\draw plot [smooth] coordinates {(0.22*#1,0.3*#1)
			 (0.27*#1,0.27*#1)
			 (0.32*#1,0.3*#1)};
		\end{scope}
	}
}

% this is for the guy looking at the front
\tikzset{
	frontVR/.pic = {
	% the only parameter is the size
		% first the body
		\draw plot [smooth] coordinates {(0,0)
			(0.08*#1,0.2*#1)
			(0.42*#1,0.2*#1)
			(0.5*#1,0)};
		\draw (0,0) -- (0.5*#1,0);
		% then the head
		\draw[fill=white] (0.25*#1, 0.4*#1) ellipse (0.15*#1 cm and 0.2*#1 cm);
		\draw plot [smooth] coordinates {(0.18*#1,0.3*#1)
			 (0.25*#1,0.27*#1)
			 (0.32*#1,0.3*#1)};
		% then the VR headset
		\draw[rounded corners=0.5*#1, fill=black] (0.07*#1, 0.55*#1) rectangle
			(0.43*#1, 0.35*#1);
	}
}


\def\sizeBig{11}
\def\sizeMed{9}
\def\ecartMed{-6 pt}
\def\sizeLow{7}
\def\ecartLow{-6 pt}
\def\ecartInterThree{-0.52}


\pic[local bounding box=leftup] at (0,0) {threerep={1}{1}{1}{2}{1}};
\pic[anchor=east, local bounding box=leftmid] at ($(-0.4,\ecartInterThree)+(leftup.east |- leftup.south)$)
	  {threerep={1}{2}{0}{-2}{-1}};
\pic[anchor=east, local bounding box=leftdown] at ($(-0.4,\ecartInterThree)+(leftmid.east |- leftmid.south)$)
	  {threerep={1}{3}{0}{0}{0}};

\pic[local bounding box=centerup] at (1,0) {threerep={0}{0}{2}{2}{1}};
\pic[anchor=east, local bounding box=centermid] at ($(-0.4,\ecartInterThree)+(centerup.east |- centerup.south)$)
	  {threerep={0}{0}{0}{-2}{-1}};
\pic[anchor=east, local bounding box=centerdown] at ($(-0.4,\ecartInterThree)+(centermid.east |- centermid.south)$)
	  {threerep={0}{0}{0}{0}{0}};

\pic[local bounding box=rightup] at (2,0) {threerep={0}{0}{3}{2}{1}};
\pic[anchor=east, local bounding box=rightmid] at ($(-0.4,\ecartInterThree)+(rightup.east |- rightup.south)$)
	  {threerep={0}{0}{0}{-2}{-1}};
\pic[anchor=east, local bounding box=rightdown] at ($(-0.4,\ecartInterThree)+(rightmid.east |- rightmid.south)$)
	  {threerep={0}{0}{0}{0}{0}};

%\begin{scope}[on background layer]
%	\draw[fill=gray!10,draw=none] ([yshift=4.5pt]leftmid.north west) rectangle
%				([xshift=3pt, yshift=-4.5pt]rightmid.south east);
%\end{scope}

% axis time
\node[inner sep=1pt] (timeleg) at ([xshift=8pt, yshift=-8pt]rightdown.south) {t};
\draw [thick, ->] ([yshift=-8pt] leftdown.south) to (timeleg);

\draw[dotted] ([xshift = 4pt]leftdown.east |- timeleg.south) to
	([xshift = 4pt]leftdown.east |- leftup.north);

\draw[dotted] ([xshift = 4 pt]centerdown.east |- timeleg.south) to
	([xshift = 4pt]centerdown.east |- centerup.north);

\draw[element] ([yshift=3pt]leftup.north west) rectangle
				 ([yshift=-3pt,xshift=3pt]rightdown.east |- timeleg.south);
\node[elementtitle,anchor=east,above=-1pt of leftup] (serverLeg) {\vphantom{pt}server};

% ===== client

% draw the guys
\def\ecartGuys{0.25}

\pic[local bounding box=leftguy, anchor=south] at (4.5, -2.59) {leftVR=1.5};
\pic[local bounding box=frontguy, right=\ecartGuys cm of leftguy.south east]  {frontVR=1.5};
\pic[local bounding box=rightguy, right=\ecartGuys cm of frontguy.south east]  {rightVR=1.5};

% draw the associated FoVs
\pic[local bounding box=leftfov, below=10 pt of leftguy.south] {fov={0.8}{0}};
\pic[local bounding box=frontfov, below=10 pt of frontguy.south] {fov={0.8}{1}};
\pic[local bounding box=rightfov, below=10 pt of rightguy.south] {fov={0.8}{2}};

% ==== draw the bandwidth

% axis
\node[font=\footnotesize, inner sep=1pt] (highpoint) at ([yshift=-3pt]leftfov.west |- leftup.north) {bw};
\draw[thick,->] ([yshift=10pt]leftfov.west |- leftguy.north) to (highpoint);
\node[font=\footnotesize, inner sep=1pt] (rightpoint) at ([yshift=10pt]rightfov.east |- rightguy.north) {t};
\draw[thick,->] ([yshift=10pt]leftfov.west |- leftguy.north) to (rightpoint);

% curve
\draw[ultra thick] plot [smooth] coordinates {([yshift=35pt]leftfov.west |- leftguy.north)
		([yshift=50pt]frontfov.west |- leftguy.north)
		([yshift=25pt]rightfov.west |- leftguy.north)
		([yshift=40pt]rightfov.east |- leftguy.north)};

% separating lines
\draw[dotted] ([xshift =0.04cm]leftfov.south east) to
	([xshift =0.04cm]leftfov.east |- highpoint);

\draw[dotted] ([xshift =0.04cm]frontfov.south east) to
	([xshift =0.04cm]frontfov.east |- highpoint);


\pic[local bounding box=leftBW] at ([yshift=18pt] leftguy.north){emptyEquirec={\sizeLow}};
\node[font=\scriptsize] at (leftBW) {\vphantom{h}low};
\pic[local bounding box=frontBW] at ([yshift=18pt] frontguy.north){emptyEquirec={\sizeBig}};
\node[font=\scriptsize] at (frontBW) {\vphantom{h}high};
\pic[local bounding box=rightBW] at ([yshift=18pt] rightguy.north){emptyEquirec={\sizeLow}};
\node[font=\scriptsize] at (rightBW) {\vphantom{h}low};

% draw the enclosing element
\draw[element] ([yshift=3pt]highpoint.west |- leftup.north) rectangle
				 ([yshift=-3pt,xshift=3pt]rightfov.south east);
\node[elementtitle, anchor=east] at (rightguy |- serverLeg) {\vphantom{pt}client};

% =============

\coordinate (client) at (highpoint.west);
\coordinate (server) at ([xshift=3pt]rightdown.east);

\tikzset{
	proto/.style={
     	-latex, thick
     }
}

\tikzset{
	protoleg/.style={
		sloped,
		inner sep=1pt,
		font=\tiny,
		above
     }
}

\tikzset{
	pics/reqresp/.style n args={2}{
	% first parameter is the req message
	% second parameter is the resp message
		code={
			\draw[proto] (0,0) to
				node[protoleg, midway] {#1} ([yshift=-10 pt]server |- 0,0);
			\draw[proto] ([yshift=-11 pt] server |- 0,0) to
				node[draw=none,midway,matrix] {#2} ([yshift=-21pt] 0,0);
			}
	}
}

\pic[local bounding box=mpd] at (client) {reqresp={connect}
	{\node[font=\tiny,inner sep=1pt,thin,draw=gray,fill=white] at (0,0) {mpd};\\}
	};
\pic[local bounding box=oneseg] at ([yshift=-3pt]client|-mpd.south) {reqresp=
	{s$_1$:QEC$_2$\,lo}
	{\pic at (0,0) {equirec={\sizeLow}{-2}{-1}};\\}
	};
\pic[local bounding box=twoseg] at ([yshift=-3pt]client|-oneseg.south)
	{reqresp={s$_2$:QEC$_3$\,hi}
	{\pic at (0,0) {equirec={\sizeBig}{0}{0}};\\}
	};
\pic[local bounding box=thirdseg] at ([yshift=-3pt]client|-twoseg.south)
	{reqresp={s$_3$:QEC$_1$\,lo}
	{\pic at (0,0) {equirec={\sizeLow}{2}{1}};\\}
	};

%\draw[proto] (client) to node[protoleg] {\pgfmathset\baisse-\traitement} ([yshift=-\baisse pt] server |-client);
%\draw[proto] ([yshift=-\baisse-\traitement pt] server |- client) to
%	node[protoleg]{mpd} ([\yshift=-\baisse - \baisse-\traitement pt] client);

\end{tikzpicture}

   \caption{Viewport-adaptive streaming system: the server offers \num{6} representations (\num{3} \acp{QER} at \num{2} bit-rates). The streaming session lasts for three segments. The client head moves from left to right, the available bandwidth varies. For each segment, the client requests a representation that matches both the \FoV{} and the network throughput.}
   \label{fig:newdelivery}
\end{figure}

\parag{Client}Over time the viewer moves the head and the
available bandwidth changes. Current \acp{HMD} record changes
in head orientation through rotation around three perpendicular axes,
denoted by \emph{pitch}, \emph{yaw}, and \emph{roll}.
Head movements modify the \FoV{} center, requiring a new viewport
to be displayed. State-of-the-art \acp{HMD} can perform the
extraction~\cite{fovhmds}. The client periodically sends a request
to the server for a new segment in the representation that
matches both the new \FoV{} center and the available throughput.

\parag{Adaptation algorithm}Similarly to \ac{DASH}, the client runs
an adaptation algorithm to select the video representation. It first
selects the \ac{QER} of the video based on the \FoV{} center and
the \acp{QEC} of the available \acp{QER}. This is an important addition to
the \ac{DASH} bit-rate adaptation logic, since the \ac{QER} determines
the quality of the video that is delivered and displayed to the user.
After the \ac{QER} selection, the client chooses the video
representation characterized by this \ac{QER} and whose bit-rate fits
with the expected throughput for the next $x$ seconds (\textit{i.e.},
$x$ being the segment length). The server replies
with the requested video representation, from which the
client extracts the viewport, displaying it on the \ac{HMD}, as
shown in Figure~\ref{fig:newdelivery}.

Rate-adaptive streaming systems are based on the assumption that
the selected representation will match the network
conditions for the next $x$ seconds. Rate adaptation algorithms
are
%developed~\cite{tian,probe_li_2014,miller,zou,liu} to reduce the mismatch between the
developed~\cite{tian,zou,liu} to reduce the mismatch between the
requested bit-rate and the throughput. In our proposal, the adaptation algorithm should also ensure
that the \FoV{} centers will be as close as possible to the \ac{QEC} of the chosen \ac{QER} during
the $x$ next seconds.
In this paper, we implement
a simple algorithm for \ac{QEC} selection: we select the \ac{QEC} that
has the smallest orthodromic distance\footnote{The shortest distance
between two points on the surface of a sphere, measured along the
surface of the sphere. Its measure is proportional to the radius
of the sphere; we refer to ``distance unit'' to denote the
radius size.} to the \FoV{} center at the time the client runs
the adaptation algorithm. Similarly as for bit-rate adaptation, we
expect new viewport-adaptive algorithms to be developed in
the future to better predict the head movement and select the
\ac{QEC} accordingly. In their recent paper,~\citet{allthings} have
made a first study where they show that a simple linear regression algorithm
enables an accurate prediction of head movements for short segment size.



\parag{Video segment length}A video segment length determines how
often requests can be sent to the server. It typically ranges from
\SIrange{1}{10}{\second}. Short segments enables quick
adaptation to head movement and bandwidth changes, but it increases
the overall number of segments and results in larger manifest files.
Shorter segments also increase the network overhead due to
frequent requests, as well as the network delay because of the round
trip time for establishing a TCP connection.
Longer segments improve the encoding efficiency and quality relative to
shorter ones, however they reduce the flexibility to adapt the video
stream to changes. We discuss segment length and head movement in
Section~\ref{subsec:segmentLength} based on a dataset.



\begin{lstlisting} [float, language=xml, frame=single, backgroundcolor=\color{white},lineskip={-1pt}, caption=Extensions of MPD file,captionpos=b, label=mpdChanges]
<?xml version="1.0"?>
<MPD>
  <Representation id="1" qec="90,60" bandwidth="9876" width="1920" height="1080" frameRate="30">
   <EssentialProperty schemeIdUri="urn:mpeg:dash:vrd:2017" value="0,0">
   <SegmentList timescale="1000" duration="2000">
   ...
  </Representation>
 </AdaptationSet>
</MPD>
</xml>
\end{lstlisting}

\parag{Extending the \ac{MPD} file} To implement the proposed
\FoV{}-adaptive video streaming, we extended a \ac{DASH} \ac{MPD}
file with new information, as illustrated in Listing~\ref{mpdChanges}.
Each representation contains the \texttt{coordinates} of its \ac{QEC}
in degrees, besides the parameters that are
already defined in the standard~\cite{iso_iec}.
Those coordinates are the two angles of the spherical coordinates of the \ac{QEC}, ranging respectively from \SIrange{0}{360} degrees and from \SIrange{-90}{90} degrees. All representations from the same adaptation set should have the same reference coordinate system.
The \texttt{@schemeIdUri} is used to indicate some extra information on the video such as the video source id and the projection type. The projection type is used by the client to determine if he knows how to extract viewports from this layout.
%It is also possible to define some \acp{RoI} to describe the geometry of the \ac{QEC} and the distribution of the quality.
%The \acp{RoI} are not mandatory in our architecture because the \ac{QEC} is often sufficient to decide which segment to select.
%If a \ac{RoI} is defined, we recommend using a \ac{RoI} defined on the sphere and not inside the projected video.
