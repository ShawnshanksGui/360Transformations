\section{Conclusion}
\label{sec:conclusion}

We have introduced in this paper viewport-adaptive streaming for
navigable 360-degree videos. \GS{Our system aims at offering both
interactive highs-quality service to \ac{HMD} users with low management
for \ac{VR} providers.
We studied the main system
settings of our framework, and validated its relevance}.
%, which result in quality increase for a given bit-rate
%budget.
\XC{We emphasize that, with current encoding techniques, the cube
map projection for two seconds segment length and six
\acp{QEC} offers the best performance}. This
paper opens various research questions:
\begin{itemize}
  \item new adaptation algorithms should be studied for viewport navigation,
especially based on head movement prediction techniques.

  \item new video encoding methods should be
developed to perform quality-differentiated encoding for
large-resolution videos. \XC{Especially, methods that allow for
intra-prediction and motion vector prediction across \emph{different} quality
areas}.
%, and methods that takes into account the statistics of 360-degree
%videos.}{}
%$(iii)$ new delivery mechanisms should be developed for the
%special case of ultra-wide-band wireless transmission at home.
\item specific studies for \emph{live} \ac{VR} streaming
and interactively-generated 360-degree videos should be performed,
because the different representations can hardly be all
generated on the fly.
\end{itemize}
