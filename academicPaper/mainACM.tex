
%\documentclass[10pt,onecolumn,twoside]{IEEEtran} %!PN
%  \documentclass[10pt,twocolumn,twoside]{IEEEtran} %!PN
% \documentclass[12pt,onecolumn,twoside,draft]{IEEEtran} %!PN
%\documentclass[conference]{IEEEtran} %!PN
%\documentclass[conference]{IEEEtran}

\documentclass{sig-alternate}
%\documentclass{article}
%\usepackage{spconf} %ICME conf style

\usepackage{multirow}
\usepackage{amsfonts}
\usepackage{epsfig}
\usepackage{amsmath}
\usepackage{amssymb}
%%\usepackage[nolist]{acronym}
\usepackage[acronym]{glossaries}                                           
\newcommand\acro[2]{\newacronym{#1}{#1}{#2}}                               
\newcommand\acroAlwaysShort[1]{\newglossaryentry{#1}{type=\acronymtype, name={#1}, description={#1}, text={#1}, first={#1}, plural={#1s}, firstplural={#1s}}}
\newcommand\acroShortSurname[2]{\newglossaryentry{#1}{type=\acronymtype, name={#2}, description={#2}, text={#2}, first={#2}, plural={#2s}, firstplural={#2s}}}
\newcommand\ac[1]{\gls{#1}}                                                
\newcommand\acp[1]{\glspl{#1}}                                             
\newcommand\acs[1]{\glsname{#1}}                                           
\newcommand\acl[1]{\glsentrylong{#1}}

\acro{FoV}{Field of View}
\acro{ABR}{adaptive bit-rate}
\acro{DASH}{Dynamic Adaptive Streaming over HTTP}  
\acro{CDN}{content delivery network}  
\acro{CDF}{cumulative density function}
\acro{HMD}{Head-Mounted Display}
\acro{PDF}{probability density function}  
\acro{RTP}{real-time protocol}
\acro{VR}{Virtual Reality}
\acro{QoE}{Quality of Experience}
\acro{VQM}{Video Quality Metric}
\acro{ILP}{integer linear program}
\acro{HDTV}{high definition television}
\acro{UGC}{User-Generated Content}
\acro{MPD}{Media Presentation Description}
\acro{PSNR}{Peak Signal Noise to Ratio}
\acro{GPU}{graphics processing unit}
\acro{CPU}{central processing unit}
\acro{MS-SSIM}{Multiscale - Structural Similarity}
\acro{API}{Application Programming Interface}
\acro{QEC}{Quality Emphasis Center}
\acro{VM}{Virtual Machine}
\acro{SVC}{Scalable Video Coding}
\acro{GOP}{Group of Picture}
\acro{PMU}{Performance Monitoring Unit}
\acro{LAN}{Local Area Network}
\acro{AI}{Artificial Intelligence}
\acro{3D}{Three Dimentional}
\acro{OS}{Operating System}
\newglossaryentry{fps}{type=\acronymtype, name={fps}, description={frame per second}, text={frame per second}, first={frame per second (fps)}, plural={fps}, firstplural={frames per second (fps)}}
\newglossaryentry{p}{type=\acronymtype, name={p}, description={p}, text={~pixel}, first={~pixel (p)}, plural={p}, firstplural={~pixels (p)}}
\newglossaryentry{s}{type=\acronymtype, name={s}, description={s},
text={second}, first={second (s)}, plural={s}, firstplural={~seconds (s)}}
\acroAlwaysShort{TCP}
\acroAlwaysShort{HTTP}
\acro{RTT}{Round-Trip Time}
\acro{AVC}{Advanced Video Coding}
\acro{HEVC}{High Efficiency Video Coding}
\acro{ISO}{International Organization for Standardization}
%\acro{ISOBMFF}{\ac{ISO} base media file format}
\newglossaryentry{ISOBMFF}{type=\acronymtype, name={ISOBMFF}, description={International Organization for Standardization base media file format}, text={International Organization for Standardization base media file format}, first={International Organization for Standardization base media file format ISO/IEC 14496-12 (ISO BMFF)}, plural={ISO BMFFs}, firstplural={International Organization for Standardization base media file formats (ISO BMFFs)}}
\acro{MTU}{Maximum Transmission Unit}
\acro{AQM}{Active Queue Management}
\acro{I}{intra-predicted}
\acro{P}{inter-predicted}
\acro{B}{bidirectional}
\acro{VQMT}{Video Quality Measurement Tool}
\acro{EPFL}{Ecole Polytechnique F\'{e}d\'{e}rale de Lausanne}
\acroShortSurname{YUV}{YUV}
\acroShortSurname{RGB}{R'G'B'}
\acro{MSE}{Mean Square Error}
\acro{CMSE}{Commulative Mean Square Error}
 % acronymes + associated packages  are defined in the acronymes.tex file
\usepackage[english]{babel}
%\usepackage{cite}
\usepackage[numbers,sort]{natbib}
\renewcommand\citet[1]{\citeauthor{#1}~\cite{#1}} % redefine the \citet command to add a ~ space between authors and []
\usepackage{color}
\usepackage[dvipsnames,table]{xcolor}
\usepackage{stfloats}
%\usepackage{pst-gantt}
% \usepackage{algorithm}
% \usepackage{algorithmic}
\usepackage[linesnumbered,ruled,vlined,boxed,commentsnumbered]{algorithm2e}
\usepackage[noend]{algorithmic}
\usepackage{float}
\algsetup{linenosize=\tiny}
%\usepackage{caption}
%\usepackage{subcaption}
\usepackage[caption=false]{subfig}
\usepackage{pgfplots}
\usepackage{booktabs}
\usepackage{listings}
\usepackage[hidelinks]{hyperref}
%\usetikzlibrary{plotmarks}
\usetikzlibrary{shapes,positioning,3d,calc}
\usetikzlibrary{decorations,decorations.pathmorphing}
\usetikzlibrary{external}
\newcommand{\externaldirectory}{latex.out/}
\tikzexternalize[prefix=\externaldirectory]
\tikzexternalize % activate!
\tikzexternaldisable
\usepackage{wrapfig}
\usepackage{enumitem}
\usepackage{url}
\usepackage{etoolbox}

\lstset{%
  backgroundcolor=\color{gray!25},
  basicstyle=\sffamily \scriptsize,
  breaklines=true
}

%SI Units
\usepackage{siunitx}
\sisetup{detect-all}

%math tools
\usepackage{mathtools}
\usepackage{stmaryrd}
\usepackage{mathrsfs}
\usepackage{amssymb}
%Some math declarations
\DeclarePairedDelimiter{\ceil}{\lceil}{\rceil}
\DeclarePairedDelimiter{\parenthesis}{(}{)}
\DeclarePairedDelimiter{\set}{\{}{\}}
\DeclarePairedDelimiter{\norm}{|}{|}
\DeclarePairedDelimiter{\integerInterval}{\llbracket}{\rrbracket}
\DeclareMathOperator*{\minimize}{minimize}

% parag
\newcommand{\parag}[1]{\vspace{5pt}\noindent\textbf{#1}.\hspace{5pt}}

%Some symbols definitions
\newcommand{\packetset}{\mathcal P}
\newcommand{\frameset}{\mathcal F}
\newcommand{\packetsubset}{\mathcal P'}
\newcommand{\framesubset}{\mathcal F'}


% fix the issue of white character in acronym package
\usepackage{etoolbox}
\makeatletter
\patchcmd\@acf{\hskip\z@}{}{}{}
\patchcmd\@acf{\hskip\z@}{}{}{}
\makeatother


\usepackage{pgfplotstable}
\pgfplotsset{compat=1.8}


\newbool{NotesActivated}
\booltrue{NotesActivated}  %comment this line to remove all user comments

%include some macro specific for this paper
\newcommand\algoFontSize{8}
\newcommand{\noteGS}[1] {\ifbool{NotesActivated}{\color{red}\{\textbf{GS:}\textit{{#1}}\}\color{black}}{}}
\newcommand{\noteXC}[1] {\ifbool{NotesActivated}{\color{YellowOrange}\{\textbf{XC:}\textit{{#1}}\}\color{black}}{}}
\newcommand{\noteAD}[1] {\ifbool{NotesActivated}{\color{OliveGreen}\{\textbf{AD:}\textit{{#1}}\}\color{black}}{}}
\newcommand{\AD}[1] {\color{OliveGreen}#1\color{black}}
\newcommand{\GS}[1] {\color{red}#1\color{black}}
\newcommand{\XC}[1] {\color{YellowOrange}#1\color{black}}


\newcommand\newpara[1]{\vspace{3pt}\noindent\textbf{#1}.\hspace{0.15cm}}
\newcommand\newsubpara[1]{\vspace{0.15cm}\noindent\textit{#1}.\hspace{0.15cm}}

%Define names of the Estimation Functions
\newcommand{\setPositive}{ % STYLE
   \text{\bf{\scriptsize+}}
}
\newcommand{\setNegative}{ % STYLE
   %\mathbb{\tiny-}
   \text{\large-}
}
\newcommand\constantParam[1]{
    {\scriptsize \{ #1 \}}
}

\newcommand\degree[0]{\ensuremath{^\circ}}

%\newcommand\R{\emph{Random}}
%\newcommand\TR{\emph{Type}}
%\newcommand\DR{\emph{Dependencies}}
%\newcommand\SP{\emph{DropSmall}}
%\newcommand\DSP{\emph{DepDropSmall}}
%\newcommand\DSM{\emph{DepDropBig}}
%\newcommand\DTSM{\emph{HybridDropBig}}
%\newcommand\DTSP{\emph{HybridDropSmall}}


\newcommand {\otoprule }{\midrule [\heavyrulewidth]}

\title{Geometric Layout for Navigable 360-Degree Video Delivery}

\tolerance=1
\emergencystretch=\maxdimen
\hyphenpenalty=10000
\hbadness=10000

\floatstyle{ruled}
\newfloat{ilp}{ht}{aux}
\floatname{ilp}{Integer Linear Program}


\newbool{doubleBlinded}
\booltrue{doubleBlinded}  %comment this line to remove all user comments

\makeatletter

\ifdoubleBlinded
\numberofauthors{1}
\else
\numberofauthors{3}
\fi

\author{ 
\alignauthor
\ifdoubleBlinded
        Paper ID 
\else
  Xavier Corbillon\\
  \affaddr{T\'{e}l\'{e}com Bretagne, IRISA, France}% \\
\alignauthor
  Alisa Devlic\\
  \affaddr{T\'{e}l\'{e}com Bretagne, IRISA, France}% \\
\alignauthor
  Gwendal Simon\\
  \affaddr{T\'{e}l\'{e}com Bretagne, IRISA, France}%\\
\fi
}



\makeatother

\usetikzlibrary{calc}
\usetikzlibrary{intersections}

%\definecolor{color1}{HTML}{73d216}
\definecolor{color1}{rgb}{0.66,0.76,0.23}
\definecolor{color2}{HTML}{c17d11}
\definecolor{color3}{HTML}{cc0000}
\definecolor{color4}{HTML}{204a87}
\definecolor{color5}{HTML}{ad7fa8}

\definecolor{tangoRed}{HTML}{cc0000}
\colorlet{titles}{color1}

\definecolor{midqualityOri}{HTML}{e9b96e}
%\definecolor{fullquality}{HTML}{8f5902}
\colorlet{midquality}{beamer@tbbrown!40!midqualityOri}
\colorlet{fullquality}{beamer@tbbrown}

\tikzset{
	spherical/.pic={
		\shade[ball color=midquality,opacity=0.40] (0,0) circle (#1 pt);
		\draw (-0.0352778*#1,0) arc (180:360:#1 pt and 0.5*#1 pt);
	    \draw[densely dashed] (-0.0352778*#1,0) arc (180:0:#1 pt and 0.5*#1 pt);
   	    \draw (0,0.0352778*#1) arc (90:270:0.5*#1 pt and #1 pt);
   	    \draw[densely dashed] (0,0.0352778*#1) arc (90:-90:0.5*#1 pt and #1 pt);
     	\draw (0,0) circle (#1 pt);
    }
}

\tikzset{
	pics/equirectangular/.style n args={3}{
		code={
			%mid-quality
			\draw[fill=midquality] (#2*0.00881945*#1-3*0.00881945*#1,#3*0.004961*#1-3*0.004961*#1)
			rectangle
			(#2*0.00881945*#1+4*0.00881945*#1,#3*0.004961*#1+4*0.004961*#1);

			%full-quality
			\draw[fill=fullquality] (#2*0.00881945*#1-0.00881945*#1,#3*0.004961*#1-0.004961*#1)
			rectangle
			(#2*0.00881945*#1+2*0.00881945*#1,#3*0.004961*#1+2*0.004961*#1);

			%grid
			\foreach \i in {-4,-3,-2,-1,0,1,2,3}{
				\foreach \j in {-4,-3,-2,-1,0,1,2,3}{
					\draw (\i*0.00881945*#1,\j*0.004961*#1) rectangle (\i*0.00881945*#1+0.00881945*#1,\j*0.004961*#1+0.004961*#1);
					}
				}
		}% code
	}% pic style
}%tikzset

\tikzset{
	cubemap/.pic={
		\draw[fill=midquality] (-0.0352778*#1,-0.006615*#1) rectangle (-0.017639*#1,0.006615*#1);
		\draw[fill=fullquality] (-0.017639*#1,-0.006615*#1) rectangle (0,0.006615*#1);
		\draw[fill=midquality] (0,-0.006615*#1) rectangle (0.017639*#1,0.006615*#1);
		\draw[fill=white] (0.017639*#1,-0.006615*#1) rectangle (0.0352778*#1,0.006615*#1);
		\draw[fill=midquality] (-0.017639*#1,0.006615*#1) rectangle (0,0.019844*#1);
		\draw[fill=midquality] (-0.017639*#1,-0.006615*#1) rectangle (0,-0.019844*#1);
%		\draw[draw=black,fill=none] (-0.0352778*#1,-0.006615*#1) rectangle (0.0352778*#1,0.006615*#1);
	}
}

\tikzset{
	pyramid/.pic={
		\draw[fill=fullquality] (-0.011759*#1,-0.006615*#1) rectangle (0.011759*#1,0.006615*#1);
		% triangle north
		\draw[fill=midquality] (-0.011759*#1,0.006615*#1)
			 -- (0.011759*#1,0.006615*#1)
			 -- (0,0.019844*#1)
			 -- cycle;
		% triangle south
		\draw[fill=midquality] (-0.011759*#1,-0.006615*#1)
			 -- (0.011759*#1,-0.006615*#1)
			 -- (0,-0.019844*#1)
			 -- cycle;
		% triangle west
		\draw[fill=midquality] (-0.011759*#1,-0.006615*#1)
			 -- (-0.011759*#1,0.006615*#1)
			 -- (-0.0352778*#1,0)
			 -- cycle;
		% triangle east
		\draw[fill=midquality] (0.011759*#1,-0.006615*#1)
			 -- (0.011759*#1,0.006615*#1)
			 -- (0.0352778*#1,0)
			 -- cycle;
	}
}

\tikzset{
	pics/losange/.style n args={3}{
		code={
			\draw[fill=#2, rotate around={#3:(0,0)}]
				(0,0)
				-- (#1, 0.75*#1)
				-- (2*#1, 0)
				-- (#1, -0.75*#1)
				--	cycle;
		}
	}
}

\tikzset{
	dodecahedron/.pic={
		\def\xshi{0}
		\pic at(\xshi,0) {losange={#1}{white}{0}};
		\pic at(\xshi,0) {losange={#1}{white}{287}};
		\pic at(\xshi+2*#1,0) {losange={#1}{midquality}{254}};
		\pic at(\xshi+1.46*#1,-1.93*#1) {losange={#1}{midquality}{0}};

		\def\xshi{2.89*#1}
		\pic at(\xshi,0) {losange={#1}{fullquality}{0}};
		\pic at(\xshi,0) {losange={#1}{fullquality}{287}};
		\pic at(\xshi+2*#1,0) {losange={#1}{midquality}{254}};
		\pic at(\xshi+1.46*#1,-1.93*#1) {losange={#1}{white}{0}};

		\def\xshi{-2.89*#1}%5.78
		\pic at(\xshi,0) {losange={#1}{midquality}{0}};
		\pic at(\xshi,0) {losange={#1}{midquality}{287}};
		\pic at(\xshi+2*#1,0) {losange={#1}{midquality}{254}};
		\pic at(\xshi+1.46*#1,-1.93*#1) {losange={#1}{white}{0}};
	}
}





\begin{document}
%\ninept

\maketitle

\begin{abstract}
The delivery and display of ultra high resolution 360-degree videos on Head-Mounted Displays (HMDs) presents a number of technical challenges. 360-degree videos are high resolution spherical videos that contain an omnidirectional view of the scene, however only a portion of this scene is displayed at a time on the user's HMD. The delivery of such videos wastes network resources since most pixels are never used. With high refresh rates, HMDs need to react on the user's head movements in less than 10 ms. This prevents dynamic adjustments of video quality at the server based on the client's feedback. Instead, an entire 360-degree video scene needs to be delivered to the client to extract the appropriate fraction of this scene. To reduce a video bitrate, while still providing an immersive experience to users, this paper proposes a view-adaptive 360-degree video streaming system. In this system the server prepares multiple video representations that differ not only by bitrate, but also by qualities of different regions. The client chooses a representation for the next segment whose bitrate fits the available throughput and the full quality region matches the viewing direction. We investigate the impact of various spherical-to-plane projections and quality arrangements on the displayed video quality, showing that the cubemap layout offers the best quality for the given bitrate budget. The evaluation with a dataset of users navigating in 360-degree videos demonstrates that segments needs to be short enough to enable frequent view switches.
\end{abstract}
%%% Local Variables:
%%% mode: latex
%%% TeX-master: "paper"
%%% End:


\section{Introduction}
\label{sec:introduction}

The popularity of navigable 360-degree video systems has grown with
the advent of omnidirectional capturing systems
and interactive displaying systems, like \acp{HMD}. However, to
deliver 360-degree video content on the Internet, the content
providers have to deal with a problem of bandwidth waste: What is
displayed on the device, which is indifferently called
\textit{\ac{FoV}} or \textit{viewport}, is only a fraction of what is
downloaded, which is an omnidirectional view of the scene.
This bandwidth waste is the price to pay for interactivity. To prevent
\emph{simulator sickness}~\cite{moss2011characteristics} and to
provide good \ac{QoE}, the vendors of \acp{HMD} recommend that the
enabling multimedia systems react to head movements as fast as the
\ac{HMD} refresh rate.
%\footnote{
%\url{https://developer.oculus.com/documentation/intro-vr/latest/concepts/bp_intro/}}
Since the refresh rate of state-of-the-art \acp{HMD} is
\SI{120}{\hertz},
%\footnote{\url{http://www.vrnerds.de/vr-brillen-vergleich/}}
the whole
system should react in less than \SI{10}{ms}. This delay constraint
prevents the implementation of traditional delivery architectures
where the client notifies a server about changes and awaits for the
reception of content adjusted at the server. Instead, in the current
\ac{VR} video delivery systems, the server sends the full $360$-degree
stream, from which the \ac{HMD} extracts the viewport in real time,
according to the user head movements. Therefore, the majority of the
delivered video stream data are not used.

Let us provide some numbers to illustrate this problem. The viewport
is defined by a device-specific viewing angle (typically
$120$ degrees), which delimits horizontally the scene from the head direction center, called \FoV{} center. To ensure a
good immersion, the pixel resolution of the displayed viewport is high,
typically $4$K ($3840\times2160$). So the
resolution of the full $360$-degree video is at least $12$K
($11520\times6480$). In addition, the immersion requires a video frame
rate on the order of the \ac{HMD} refresh rate, so typically around
\SI[mode=text]{100}{\acp{fps}}. Overall, high-quality $360$-degree
videos combine both a very large resolution (up to $12$K) and a very
high frame rate (up to \SI[mode=text]{100}{\acp{fps}}). To compare,
the bit-rate of 8K videos at \SI[mode=text]{60}{\acp{fps}} encoded
using \ac{HEVC} is around \SI{100}{Mbps}~\cite{7398367}.

We propose in this paper a solution where, following the same
principles as in rate-adaptive streaming technologies, the server
offers multiple \emph{representations} of the same $360$-degree video.
But instead of offering representations that only differ by their
bit-rate, the server offers here representations that differ by having
a better quality in a given region of the video. Our proposal is a
\emph{viewport-adaptive streaming system} and is depicted in
Figure~\ref{fig:deliverychain}. Each video representation is characterized
by a \emph{\ac{QEC}}, which represents a given viewing position in the
spherical video. Around the \ac{QEC}, the quality of the video is
maximum, while it is lower for video parts that are far from the
\ac{QEC}. Similarly as in \ac{DASH}, the video is cut into segments
and the client periodically runs an \emph{adaptive algorithm} to
select a representation for the next segment. In a
viewport-adaptive system, clients select the representation
such that the bit-rate fits their receiving
bandwidth and the \ac{QEC} is closest to their \FoV{} center.

\begin{figure}
   \centering
   \begin{figure}[h]
\centering
\begin{tikzpicture}


\tikzset{
     element/.style={
     	rounded corners,
     	rectangle,
  	 	thick,
  	 	draw=black,
  	 	minimum height=2cm,minimum width=2.5cm
     }
}

\tikzset{
	elementtitle/.style={
		rectangle,
		rounded corners,
		fill=gray!80,
		font=\footnotesize,
		text=white,
		anchor=north
	}
}

\tikzset{
	pics/equirec/.style n args={3}{
		code={
			\draw[fill=gray!30] (-0.0352778*#1, -0.019844*#1) rectangle (0.0352778*#1, 0.019844*#1);
			\draw[draw=none,fill=gray!70] (0.0088194*#2*#1-2*0.0088194*#1, 0.0066147*#3*#1 - 2*0.0066147*#1) rectangle (0.0088194*#2*#1 + 2*0.0088194*#1, 0.0066147*#3*#1 + 2*0.0066147*#1);
			\draw[draw,fill=none] (-0.0352778*#1, -0.019844*#1) rectangle (0.0352778*#1, 0.019844*#1);
			\draw[color=black,fill=black] (0.0088194*#2*#1, 0.0066147*#3*#1) circle (1pt);
		}		
	}
}

\def\convCmPt{0.0352778}
\def\convCmPtRec{0.019844}
\def\convCmPtRecThird{0.0066147}
\def\convCmPtFourth{0.0088194}

\tikzset{cross/.style={cross out, draw,
         minimum size=2*(#1-\pgflinewidth),
         inner sep=0pt, outer sep=0pt}}

\tikzset{
	fov/.pic ={
		\draw[densely dotted, thick, red!70!black] (-0.07,0.10) rectangle (0.37,-0.20);
%		\draw[fill=red] (0.2,-0.05) circle (2pt);
		\draw (0.15,-0.05) node[cross=2pt,red!70!black] {};
	}
}


\tikzset{
	vr/.pic = {
		\draw[rounded corners] (-0.0352778*#1, -0.019844*#1) rectangle (0.0352778*#1, 0.019844*#1);
		\draw[rounded corners, thick] (-0.032*#1, -0.019844*#1) rectangle (0.032*#1, 0.016*#1);
%		\draw(-0.019*#1,0) pic {fov};
%		\draw(0.019*#1,0) pic {fov};
		\node[font=\scriptsize,rectangle,red, draw=red, thick,
					densely dotted, anchor=east, inner sep=2pt,
					yshift=-1pt, xshift=-1pt] at (0,0) {L};
		\node[font=\scriptsize,rectangle,red, draw=red, thick,
					densely dotted, anchor=west, inner sep=2pt,
					yshift=-1pt,xshift=0.5pt] at (0,0) {R};
	}
}
		
\def\ecartElement{20pt}
\def\sizeSphere{11}
\def\ecartObjet{2}
\def\ecartYVersions{16pt}


% capturing system
\node[element] (0,0) (capturing) {};
\node[elementtitle, above=-5pt of capturing] {capturing};
\draw ([xshift=-\sizeSphere - \ecartObjet pt]capturing.east) pic {spherical=\sizeSphere};
\pgfdeclareimage[width=18 pt]{camera}{video-camera-icon-hi.png}
\node at ([xshift=21 pt]capturing.west) (camera1)
    {\pgfbox[right,center]{\pgfuseimage{camera}}};

% the arrow in the capture
\draw[-latex] (camera1.east) to ([xshift=-2*\sizeSphere - 2*\ecartObjet pt]capturing.east);


% ===== server
\node[element,right=\ecartElement of capturing] (server) {};
\node[elementtitle, above=-5pt of server] {\vphantom{pt}server};
\draw ([xshift=\sizeSphere + \ecartObjet pt]server.west) pic {spherical=\sizeSphere};
\draw ([xshift=-\sizeSphere - \ecartObjet pt]server.east) pic {equirec={\sizeSphere}{2}{1}};
\draw ([xshift=-\sizeSphere - \ecartObjet pt, yshift=\ecartYVersions]server.east) pic {equirec={\sizeSphere}{-2}{-1}};
\draw ([xshift=-\sizeSphere - \ecartObjet pt, yshift=-\ecartYVersions]server.east) pic {equirec={\sizeSphere}{0}{0}};

% the three arrows in the server
\draw[-latex] ([xshift=2*\sizeSphere + 2*\ecartObjet pt]server.west) to ([xshift=-2*\sizeSphere - 2*\ecartObjet pt]server.east);
\draw[-latex] ([xshift=2*\sizeSphere + 2*\ecartObjet pt]server.west) to ([xshift=-2*\sizeSphere - 2*\ecartObjet pt, yshift=\ecartYVersions]server.east);
\draw[-latex] ([xshift=2*\sizeSphere + 2*\ecartObjet pt]server.west) to ([xshift=-2*\sizeSphere - 2*\ecartObjet pt, yshift=-\ecartYVersions]server.east);

% between capture and server
\draw[-latex] ([xshift=2pt]capturing.east) to ([xshift=-2pt]server.west);


% ===== client
\node[element,right=\ecartElement of server] (client) {};
\node[elementtitle, above=-5pt of client] {\vphantom{pt}client};
% the equirecntagular
\draw([xshift=\sizeSphere + \ecartObjet pt, yshift=-\ecartYVersions]client.west) pic {equirec={\sizeSphere}{0}{0}};
% the old vr
%\draw ([xshift=-\sizeSphere - \ecartObjet pt]client.east) pic {vr=\sizeSphere};
% the fov
\pic[local bounding box=thisfov] at ([xshift=\sizeSphere + \ecartObjet pt, yshift=-\ecartYVersions]client.west) {fov};
%\node[anchor=west, font=\tiny,red, text width=23pt,align=center]
%		at ([yshift=10pt]client.west) (fovleg) {\ac{FoV} and \ac{FoV} center};
%
%\draw[red] (fovleg) -- (thisfov);

% vr headset
\pgfdeclareimage[width=24 pt]{vrheadset}{vr_icon.png}
\node at ([xshift=-28 pt]client.east) (headset)
    {\pgfbox[left,center]{\pgfuseimage{vrheadset}}};

% old vr
%

% the arrow in the client
\draw[-latex] 
%	([xshift=2*\sizeSphere + 2*\ecartObjet pt, yshift=-\ecartYVersions]client.west) to 
	(thisfov.60) to
		([xshift=-2*\sizeSphere - 2*\ecartObjet pt]client.east);


% between server and client
\draw[-latex] ([xshift=2pt, yshift=-\ecartYVersions]server.east) to ([xshift=-2pt, yshift=-\ecartYVersions]client.west);


\end{tikzpicture}
\caption{\ac{FoV}-adaptive 360-degree video delivery system: The server
offers video representations for three \acp{QEC}. The dark grey is the part of the video encoded at high quality and the light
gray the low quality. The \ac{FoV} video is the dotted red rectangle, and the \ac{FoV} center is the
cross}
\label{fig:deliverychain}
\end{figure}

   \caption{Viewport-adaptive 360-degree video delivery system: The server
   offers video representations for three \acp{QEC}. The dark \ifbool{withColor}{brown}{gray} is the part of the video encoded at high quality, the light
   \ifbool{withColor}{brown}{gray} the low quality. The viewport is the dotted red rectangle, the \FoV{} center the
   cross}
   \label{fig:deliverychain}
\end{figure}

This viewport-adaptive $360$-degree streaming system has three
advantages: $(i)$ the bit-rate of the delivered video is lower than
the original full-quality video because video parts distant from the
\ac{QEC} are encoded at low quality. $(ii)$ When the end-user does not
move, the viewport is extracted from the highest quality part of the
spherical video. And $(iii)$ when the head of the end-user moves, the
device can still extract a viewport because it has the full
spherical video. If the new \FoV{} center is far from the \ac{QEC}
of the received video representation, the quality of the
extracted viewport is lower but this degradation holds only until the
selection of another representation with a closer \ac{QEC}.

The remainder of the paper is organized as follows. First, we
present our viewport-adaptive streaming
system, and
we show how it can be integrated into
the \ac{MPEG} \ac{DASH}-VR standard. Our proposal is thus a contribution
to the \ac{VR} group that \ac{MPEG} launched in May
$2016$~\cite{mpeg-vr}. Second, we address the choice of the geometric
layout into which the spherical video is projected for
encoding. We evaluate several video quality arrangements for a given
geometric layout and show that the cube map layout with full quality around the \ac{QEC} and \SI{25}{\percent} of this quality in the remaining faces offers the best quality of the extracted viewport.
Third, we study the required video segment length for
viewport-adaptive streaming. Based on a dataset of real users
navigating $360$-degree videos, we show that head movements occur over
short time periods, hence the streaming video segments have to be
short enough to enable frequent \ac{QEC} switches. Fourth, we
examine the impact of the number of \acp{QEC} on the viewport quality
and we show that a small number of (spatially-distributed over the sphere)
\acp{QEC} suffices to get high viewport quality.
Finally, we introduce a tool (released as open source), which creates video representations for the proposed viewport-adaptive streaming system.
The tool is highly configurable: from a given
$360$-degree video, it allows any arrangement of video quality for a
given geometric layout, and it extracts the viewport from any \FoV{} center.
This tool thus provides the main software module
for the implementation of viewport-adaptive streaming of navigable
$360$-degree videos.

%%% Local Variables:
%%% mode: latex
%%% TeX-master: "paper"
%%% End:


% =============
\section{Background}

In this Section, we provide the background for our study.
First, we depict the overall architecture of the delivery system.
Then, we recall the main 2D geometric layouts for spherical videos.

\subsection{Navigable 360-degree Video Delivery}

The principles of a navigable video delivery system are similar as in adaptive bit-rate
video systems such as \ac{DASH}. The server offers multiple versions of the same video 
and the client
selects the most appropriate version according to some criteria. These versions 
are cut into second-long segments such that the client can regularly switch from one 
version to another.

In the case of wide video with different spatial qualities, the main idea is to spatially cut 
the video into \emph{tiles}.
Then, two implementations are possible for the delivery system. Either the server 
offers each tile at different qualities. The client selects each tile version independently 
and it has to reconstruct
the full video from these tiles before the \ac{FoV} extraction. This solution allows a 
fine setting of qualities but 
most of the computation is done at the client side. The second option, 
which is the one we consider in this paper and is also the industrial implementation described 
by~\citet{facebook}, is that
the server prepares $x$ versions related to $x$ different \acp{QEC}. Each version 
is an arrangement of tile
qualities such that the tiles that are close to the \ac{QEC} are at high-quality 
and the other tiles
are at a lower quality. The main advantages include an easy management of the server 
(\textit{e.g.} small \emph{manifest} file), a simple selection process for the client (by
a distance computation), and no need of re-constructing the video before the \ac{FoV} extraction.

At the client side, the end-user moves its head to decide the \ac{FoV}. The head movements
are called \emph{yaw}, \emph{pitch}, and \emph{roll}. The center of the \ac{FoV} is a 
point on the sphere, the size of the \ac{FoV} depends on the device (typically
around 100$^\circ$ in state-of-the-art devices), and the orientation of the extracted video 
is related to the roll.

The complete analysis and evaluation of the navigable 360-degree video delivery system
is left for future work. Due to lack of space, we focus here on the geometric layout
of the video versions and the tile quality arrangement.


\subsection{Geometric Layouts for Spherical Videos}

The projection of a sphere into a plane (known as mapping) has been extensively studied
for centuries. In this paper, we consider the four projections that are the most natural
candidates for 360-degree video delivery. These are depicted in Figure~\ref{fig:mapping}.

\begin{figure}[ht]
\centering
\begin{tikzpicture}
\def\sizeSphere{20}%pt
\def\ecartY{-1.2}%cm
\def\ecartX{6}

% da sphere
\pic [local bounding box=spher]  at (0,0) {spherical=15};

% recantagular
\pic [local bounding box=equi] at (-3,\ecartY) {equirectangular={\sizeSphere}{-1}{0}};

% cupe map
\pic [local bounding box=cubemap] at (-1,\ecartY) {cubemap=\sizeSphere};

% pyramid
\pic [local bounding box=pyra] at (1,\ecartY) {pyramid=\sizeSphere};

% rhombic
%\pgfdeclareimage[width=36 pt]{dodecahedron}{RhombicDodecahedron.png}
%\node at (3,\ecartY) (dodec)
%    {\pgfbox[center,center]{\pgfuseimage{dodecahedron}}};
    
\def\unitused{0.22}

\pic [local bounding box=dodeca] at (3,0.88*\ecartY) {dodecahedron=\unitused};

% links
\draw[-latex] (spher.180) -| (equi);
\draw[-latex] (spher.200) -| (cubemap);
\draw[-latex] (spher.340) -| (pyra);
\draw[-latex] (spher) -| (dodeca);

\node[font=\scriptsize,anchor=north] at (equi.south) {equirectangular};
\node[font=\scriptsize,anchor=north] at (cubemap.south) {cube map};
\node[font=\scriptsize,anchor=north] at (pyra.south) {pyramid};
\node[font=\scriptsize,anchor=north] at (dodeca.south) {\vphantom{y}dodecahedron};

\end{tikzpicture}
\caption{Projections into four geometric layouts}\label{fig:mapping}
\end{figure}

The advantages and shortcomings of every projection have been studied in the literature. Shortly put, from
the images that are
projected on a equirectangular, a cube map, and a rhombic dodecahedron, it is possible to generate a \ac{FoV}
for any position and angle in the sphere. Indeed, no information is loss, because no pixel is under-sampled in 
this projection (no pair of pixels in the sphere is projected on these geometrical layouts in only one pixel). However,
some pixels from the spherical image are over-sampled in the projected image. It is typically the case for
the equirectangular layout, for which the projection generates a significant over-sampling at the poles. On the 
contrary, the pyramid is not a geometric layout for lossless projections. Some pixels (those who are in the back
of the view face) are under-sampled, so two pixels can be projected into one pixel by interpolating their color
values. A \ac{FoV} that is extracted for positions near the back can suffer from distortion.

\section{Geometric Layout Evaluation}

Our objective is to identify the layout that offers the best performance regarding the quality of
the extracted \ac{FoV} video and the bit-rate of the delivered 360-degree video. We first introduce
the tool that we release to generate and exploit quality-differentiated
360-degree videos on geometric layouts. Then, we describe the testbed that we set up to evaluate the
performance of the four main geometric layouts. Finally, we present our first results.

\subsection{Our Tool to Project 360-degree Video Into Quality-Differentiated Layouts}

The code and all the details are available at \textit{hidden url}. The main features of this tool include:
\begin{itemize}
\item Projection from a spherical video into any of the four geometric layouts and vice versa. Our tool is in
particular able to re-project the 360-degree videos that are encoded and stored in the equirectangular layout,
which corresponds to a large fraction of current catalogs of 360-degree videos.
\item Adjusting video qualities for each geometric face of any layout by setting \AD{a resolution of the face in the number of pixels. In the cubemap and rhombic dodecahadron the edge length of each face is specified, while in the pyramid layout a height of the pyramid is adjusted to yield the desired resolution. The equirectangular layout enables resolution of each tile to be defined.}{}
\item \ac{FoV} extraction for any spherical coordinate and any angle, in particular for quality-differentiated videos.
\end{itemize}

The software can thus be used by the scientific community to study geometric layouts, quality arrangement over the tiles, and \ac{FoV} extraction strategies.

\subsection{Testbed Description}

The testbed that we
set up is depicted in Figure~\ref{fig:testbed}. From the original spherical video, we considered the
projection
into the four geometric layouts (equirectangular, cube map, pyramid, dodecahedron). For
each layout, we built
$x$ video versions characterized by different \acp{QEC}. Then we simulated
a client watching in a specific direction. We selected, for each layout, the version such that the \ac{QEC}
is closest to the given direction. We extracted the \ac{FoV} from this version. Finally, we compared
the quality of four different \ac{FoV} videos against the reference video generated
without quality loss from the spherical video.

\tikzsetnextfilename{testbed}
\begin{tikzpicture}

\def\ecartx{1}
\def\ecartybig{0.5}
\def\ecartysmall{0.3}
\def\heimax{8*\ecartysmall+3*\ecartybig}
\def\sizeIcon{12}
\def\unitdodeca{0.11}

\pic [local bounding box = equiFirst] at (\ecartx,0.5*\heimax)
		{equirectangular={\sizeIcon}{-1}{0}};
\pic [local bounding box = equiSecond,
		below= \ecartysmall of equiFirst] {equirectangular={\sizeIcon}{0}{-1}};
%\pic [local bounding box = equiThird,
%		below=\ecartysmall of equiSecond] {equirectangular={\sizeIcon}{0}{0}};

\pic [local bounding box = cubeFirst, below=\ecartybig of equiSecond] {cubemap=\sizeIcon};
\pic [local bounding box = cubeSecond,
		below=\ecartysmall of cubeFirst] {cubemap=\sizeIcon};
%\pic [local bounding box = cubeThird,
%			below=\ecartysmall of cubeSecond] {cubemap=\sizeIcon};

\pic [local bounding box = pyraFirst, below=\ecartybig of cubeSecond] {pyramid=\sizeIcon};
\pic [local bounding box = pyraSecond,
		below=\ecartysmall of pyraFirst] {pyramid=\sizeIcon};
%\pic [local bounding box = pyraThird,
%			below=\ecartysmall of pyraSecond] {pyramid=\sizeIcon};

\pic [local bounding box = dodecaFirst, below=\ecartybig of pyraSecond,
		xshift=-3pt] {dodecahedron=\unitdodeca};
\pic [local bounding box = dodecaSecond, below=\ecartysmall of dodecaFirst, xshift=-6pt,
		yshift=2pt] {dodecahedron=\unitdodeca};
%\pic [local bounding box = dodecaThird,
%			below=\ecartysmall of dodecaSecond] {dodecahedron=\unitdodeca};

% ==== sphere
\pic [local bounding box=spher, yshift=10pt,
		right=-\ecartx cm of pyraFirst] at (0,0) {spherical=10};

% == links
\draw[-latex] (spher) to ([xshift=-3pt, yshift=5pt]equiSecond.west);
\draw[-latex] (spher) to ([xshift=-3pt, yshift=5pt]cubeSecond.west);
\draw[-latex] (spher) to ([xshift=-3pt, yshift=5pt]pyraSecond.west);
\draw[-latex] (spher) to ([xshift=-3pt, yshift=5pt]dodecaSecond.west);

% ==== selected versions
\pic [local bounding box = chosenEqui,
		right= \ecartx of equiFirst] {equirectangular={\sizeIcon}{-1}{0}};
\pic [local bounding box = chosenCube, right=\ecartx of cubeFirst] {cubemap=\sizeIcon};
\pic [local bounding box = chosenPyra,
		right=\ecartx of pyraSecond] {pyramid=\sizeIcon};
\pic [local bounding box = chosenDodeca, right=\ecartx of dodecaFirst,
		xshift=-5pt, yshift=3.5pt] {dodecahedron=\unitdodeca};

% == links
\draw[-latex] ([xshift=1.5pt]equiFirst.east) to ([xshift=-1pt]chosenEqui.west);
\draw[-latex] ([xshift=1.5pt]cubeFirst.east) to ([xshift=-1pt]chosenCube.west);
\draw[-latex] ([xshift=1.5pt]pyraSecond.east) to ([xshift=-1pt]chosenPyra.west);
\draw[-latex] ([xshift=0.5pt]dodecaFirst.east) to ([xshift=-1pt]chosenDodeca.west);

% == legend
\tikzset{
	legendNode/.style={
		font=\scriptsize,
		text width=1cm,
		align=center,
		anchor=south
	}
}
\node[legendNode] at (chosenEqui |- equiFirst.north) {selected version};
\node[legendNode] at (equiFirst.north) {offered versions};

% ==== fov

\tikzset{
	pics/equirec/.style n args={4}{
		code={
			\draw[draw=none,fill=gray!30] (-0.0352778*#1, -0.019844*#1) rectangle (0.0352778*#1, 0.019844*#1);
			\draw[draw=none,fill=gray!70] (0.0088194*#2*#1-#4*0.0088194*#1, 0.0066147*#3*#1 - #4*0.0066147*#1) rectangle (0.0088194*#2*#1 + #4*0.0088194*#1, 0.0066147*#3*#1 + #4*0.0066147*#1);
			\draw[draw, very thick, densely dotted,fill=none] (-0.0352778*#1, -0.019844*#1) rectangle (0.0352778*#1, 0.019844*#1);
		}
	}
}

\pic [local bounding box=fovEqui, right=\ecartx of chosenEqui] {equirec={\sizeIcon}{-2}{-1}{2}};
\pic [local bounding box=fovCube, right=\ecartx of chosenCube] {equirec={\sizeIcon}{1}{0}{3}};
\pic [local bounding box=fovPyra, right=\ecartx of chosenPyra] {equirec={\sizeIcon}{0}{-1}{2}};
\pic [local bounding box=fovDodeca] at (fovPyra |- chosenDodeca) {equirec={\sizeIcon}{0}{0}{3}};

\node[legendNode] at (fovEqui |- equiFirst.north) {\ac{FoV} extraction};

% == links

\draw[-latex] ([xshift=1.5pt]chosenEqui.east) to ([xshift=-1pt]fovEqui.west);
\draw[-latex] ([xshift=1.5pt]chosenCube.east) to ([xshift=-1pt]fovCube.west);
\draw[-latex] ([xshift=1.5pt]chosenPyra.east) to ([xshift=-1pt]fovPyra.west);
\draw[-latex] ([xshift=0.5pt]chosenDodeca.east) to ([xshift=-1pt]fovDodeca.west);

% ==== reference video
\tikzset{
	refVideo/.pic={
		\draw[draw=black, very thick, densely dotted, fill=gray!70] (-0.0352778*#1, -0.019844*#1) rectangle (0.0352778*#1, 0.019844*#1);
	}
}

\pic[local bounding box=fovRef,yshift=-15pt]
		at (fovDodeca |- dodecaSecond.south) {refVideo=\sizeIcon};
\node[legendNode, text width=2cm] at (fovRef.north) {reference video};

\draw[-latex] (spher.south) |- (fovRef.west);

% ==== comparison MS-SSIM

\node[rounded corners,rectangle,thick,fill=gray!10,
		text width=1.5cm,align=center,anchor=west] (msssim)
		at ([xshift=\ecartx cm]fovEqui |- spher) {MS-SSIM comparison};

% == links

\draw[-latex] (fovEqui.east) -| (msssim.45);
\draw[-latex] (fovCube.east) -| (msssim.135);
\draw[-latex] (fovPyra.east) -| (msssim.225);
\draw[-latex] (fovDodeca.east) -| (msssim.270);
\draw[-latex] (fovRef.east) -| (msssim.315);

\end{tikzpicture}


To generate the different videos, we made some choices, which are mostly in conformance to the
literature and to real implementation of 360-degree video delivery systems. Our future work
includes to study in more details the impact of some of these choices on
the overall performance of the system; it is not our objective here.

We generated $x=16$ different video versions for each layout. This number of versions is smaller than
the number of videos (with pyramid layouts) given by~\citet{facebook}. It is however closer to the
number of video representations that are recommended in rate-adaptive streaming
systems~\cite{Aparicio-PardoP15}. We describe now specific quality arrangements per layout:
%\begin{description}
%\item[equirectangular]

\parag{Equirectangular}We cut the equirectangular layout into $8\times 8$ tiles as proposed in
the literature related to panorama video~\cite{gaddam_tiling_2015}. Tiles are thus rectangular.
We considered three different qualities: the \emph{full quality}, which corresponds to the quality of the
input spherical video, a \emph{medium quality}, which is set as half as good as the full quality, and a
\emph{low quality}, which is one quarter of the full quality. The quality arrangement is done such that
the $3\times 3$ tiles that are around the \ac{QEC} are full quality, the $7\times 7$ tiles around
the \ac{QEC} (excluding the closest $3\times 3$) are with the medium quality, and the remaining tiles
are with the low quality.

\parag{Cube Map}To generate the $x=16$ \acp{QEC}, we rotate the cube in the sphere so that the
center of the square front face matches the \ac{QEC}. The front face is at full quality, while the left, right,
top, and bottom faces are mid quality and the back face is low quality.

\parag{Pyramid}This layout differs from other layouts. First, it does not preserve
the pixel information of the spherical videos. Second, the distortion depends on the size of the square
base face and on the distance from the peak. As for the cube map, we rotate the pyramid to adjust
the center of the base face to the \ac{QEC}. The base is at full quality while the other faces are
at a medium quality. \noteGS{To be written}

\parag{Rhombic Dodecahedron}We rotated the dodecahedron such that the \ac{QEC} is between
the two same faces (say face\,1 and face\,2). Both faces are at full quality. Then, the eight
faces that are around
these faces are at medium quality, and the two remaining faces are at low quality.

We observed that, for the quality setting that we decided for the equirectangular (medium quality and
low quality at half and a quarter quality of the full quality respectively), the bit-rate of the
generated equirectangular layouts are larger than for the other layouts. To get the
same bit-rate for each video version, we apply the following process. We set that, for a given
360-degree video, the projection on the equirectangular provides the \emph{bit-rate budget}. For the
other layout, the low quality is kept at a quarter of the full quality, but we select the video quality of the medium quality such that the bit-rate of the generated video version is equal to the
bit-rate budget.

Other settings are as follows.
The 360-degree video that we chose is \noteGS{To be given}. The encoding of the video in each
geometric layout is done by HEVC \noteGS{To be given}. The direction of the client heads is as follows.
First we chose a \ac{FoV} center by a uniform random choice. Then we selected a second \ac{FoV} center
by a random choice that increases the chances to select a position near the equator. We consider
both points as the origin and the destination \ac{FoV} centers. The selection of the selected version
is based on the distance between the origin \ac{FoV} center and the \ac{QEC}. To compare the
quality of the \ac{FoV} video against the reference video, we use the \emph{MS-SSIM} quality
evaluation tool.


\subsection{Results}
\label{subsec:results}

\begin{figure}
    \usetikzlibrary{pgfplots.statistics}
\newcommand\storelabel[2]{\expandafter\xdef\csname label#1\endcsname{#2}}
\newcommand\getlabel[1]{\csname label#1\endcsname}

\newcommand{\newBoxplot}[5]{%
    \addplot+[ boxplot prepared={%
            lower whisker={#1}, lower quartile={#2}, median={#3},
            upper quartile={#4}, upper whisker={#5},% average={#6},
%            box extend=0.5,  % height of box
%            whisker extend=0.5, % height of whiskers
%            every box/.style={thin,draw=black},
%            every whisker/.style={black,thick},
%            every median/.style={black,thick},
%            every average/.style={draw=red, /tikz/mark=* },
    /pgf/number format/precision=2 } ]
    coordinates {}
    %node[left,black] at
    %(boxplot box cs: \boxplotvalue{lower whisker},0.5)
    %{\tiny\pgfmathprintnumber{\boxplotvalue{lower whisker}}}
    %node[right,black] at
    %(boxplot box cs: \boxplotvalue{upper whisker},0.5)
    %{\tiny\pgfmathprintnumber{\boxplotvalue{upper whisker}}}
    ;
} %end of \newBoxplot definition

\newcommand{\newBoxPlotFromCdfFile}[2]{%
%    \pgfplotstabletypeset[ignore chars={\#}]{#1}
    \pgfplotstablegetelem{0}{#2}\of{#1}
    \edef\min{\pgfplotsretval}
    \storelabel{min#1#2}{\min}
    \pgfplotstablegetelem{50}{#2}\of{#1}
    \edef\qmin{\pgfplotsretval}
    \storelabel{qmin#1#2}{\qmin}
    \pgfplotstablegetelem{100}{#2}\of{#1}
    \edef\med{\pgfplotsretval}
    \storelabel{med#1#2}{\med}
    \pgfplotstablegetelem{150}{#2}\of{#1}
    \edef\qmax{\pgfplotsretval}
    \storelabel{qmax#1#2}{\qmax}
    \pgfplotstablegetelem{200}{#2}\of{#1}
    \edef\max{\pgfplotsretval}
    \storelabel{max#1#2}{\max}
    %\newBoxplot{\pgfmathparse{\getlabel{min#1#2}/\MaxNbFrame}\pgfmathresult}{\pgfmathparse{\getlabel{qmin#1#2}/\MaxNbFrame}\pgfmathresult}{\pgfmathparse{\getlabel{med#1#2}/\MaxNbFrame}\pgfmathresult}{\pgfmathparse{\getlabel{qmax#1#2}/\MaxNbFrame}\pgfmathresult}{\pgfmathparse{\getlabel{max#1#2}/\MaxNbFrame}\pgfmathresult}
    \newBoxplot{\getlabel{min#1#2}}{\getlabel{qmin#1#2}}{\getlabel{med#1#2}}{\getlabel{qmax#1#2}}{\getlabel{max#1#2}}
}


\pgfplotscreateplotcyclelist{My color list}{%
    {black!50,solid,thick},%
    {black!85,solid,thick}%
}

\def\ymin{0.7}

\begin{tikzpicture}
    \begin{axis}[
            boxplot/draw direction=y,
            ylabel={\acs{MS-SSIM}},
            width=1.05\linewidth,
            height=0.5\linewidth,
            boxplot={
                draw position={1/3 + floor(0.01+\plotnumofactualtype/2) + 1/3*mod(\plotnumofactualtype,2)},
                box extend=0.30,
            },
            %x=2cm,
            xtick={0,1,2,...,10},
            x tick label as interval,
            xticklabels={%
                {{\tiny Good~Bad}},%
                {{\tiny Good~Bad}},%
                {{\tiny Good~Bad}},%
                {{\tiny Good~Bad}},%
            },
            x tick label style={
                text width=2.5cm,
                align=center
            },
            cycle list name={My color list},
            legend cell align=left,
            xmin = 0,
            xmax = 4,
            ymin = \ymin,
            ymax = 1,
            axis on top,
        ]
        \pgfplotsextra{\begin{scope}[on layer=axis background]
                \draw[fill=gray!14,draw=gray!14] (axis cs:1,\ymin) rectangle
                (axis cs:2,1);
                \draw[fill=gray!14,draw=gray!14] (axis cs:3,\ymin) rectangle
                (axis cs:4,1);
            \end{scope}
        }
        \newBoxPlotFromCdfFile{results/cdfQuality.csv}{goodEquirectangularTiled}
        \newBoxPlotFromCdfFile{results/cdfQuality.csv}{badEquirectangularTiled}

        \newBoxPlotFromCdfFile{results/cdfQuality.csv}{goodCubMap}
        \newBoxPlotFromCdfFile{results/cdfQuality.csv}{badCubMap}

        \newBoxPlotFromCdfFile{results/cdfQuality.csv}{goodPyramidal}
        \newBoxPlotFromCdfFile{results/cdfQuality.csv}{badPyramidal}

        \newBoxPlotFromCdfFile{results/cdfQuality.csv}{goodRhombicDodeca}
        \newBoxPlotFromCdfFile{results/cdfQuality.csv}{badRhombicDodeca}

    \end{axis}
    \begin{axis}[
            width=1.05\linewidth,
            height=0.5\linewidth,
            xmin=0,xmax=4,
            ymin=\ymin,ymax = 1,
            axis x line*=top,
            axis y line=none,
            enlargelimits=false,
            hide y axis,
            xtick={0,1,2,...,10},
            x tick label as interval,
            xticklabels={%
                { EquiTiled },%
                { Cube },%
                { Pyramid },%
                { Dodeca },%
            },
            x tick label style={
                text width=2.5cm,
                align=center
            },
        ]
    \end{axis}

\end{tikzpicture}

    \caption{A caption}
    \label{fig:box_plot}
\end{figure}

\begin{figure}
    \tikzsetnextfilename{distance_quality}
\begin{tikzpicture}
   \pgfplotscreateplotcyclelist{My color list}{%
       {color1,solid, very thick},%
       {color2, dashed, very thick},%
       {color3, densely dashed, very thick},%
       {color4, densely dotted, very thick},%
   }

   \pgfplotsset{every axis legend/.append style={
           at={(-0.03,0.97)},
   anchor=south west,
   draw=none,
   fill=none,
   legend columns=4,
   column sep=6pt,
   %font = \scriptsize,
   /tikz/every odd column/.append style={column sep=0cm},
   %font=\tiny
   }}
   \pgfplotsset{grid style={dashed,gray}}
   \pgfplotsset{minor grid style={dotted,red!20!gray}}
   \pgfplotsset{major grid style={dotted,green!50!black}}

    \begin{axis}[
            ylabel={MS-SSIM},
            xlabel={Orthodromic distance},
            width=0.95\linewidth,
            height=0.5\linewidth,
            cycle list name={My color list},
            legend cell align=left,
            xmin = 0,
            xmax = 3.14,
            ymin = 0.88,
            ymax = 1,
            ymajorgrids,
            yminorgrids,
        ] 
         \pgfplotsextra{\begin{scope}[on layer=axis background]
                \draw[draw=color5] (axis cs:0,0.955777) -- (axis cs:3.14,0.955777);
                \node[rounded corners, fill=color5,
                			font=\tiny, inner sep=2pt,
                			anchor=west, text=white] at (axis cs:0,0.955777) {\textit{\vphantom{lj}uniEqui}};
            \end{scope}
            \begin{scope}[on layer=axis background]
                   \draw[draw=black, dashed] (axis cs:1.57,0) -- (axis cs:1.57,1);
                   \node[anchor=north west,xshift=2pt,yshift=-2pt,inner sep=2pt,legendinfigure] at (axis cs:1.57, 1) {\footnotesize{During a 2\,s segment, 86\,\% of the users never explore this zone}};
               \end{scope}
        }

        \only<.(1)->{\legend{Equirec,CubeMap,Pyramid,Dodeca}}
        \only<+->{\addplot+ table [x=distance, y=qualityEquirectangularTiledLower]{results/distanceQuality.csv};}
        \only<+->{\addplot+ table [x=distance, y=qualityCubeMapLower]{results/distanceQuality.csv};}
        \only<+->{\addplot+ table [x=distance, y=qualityPyramidLower]{results/distanceQuality.csv};}
        \only<+->{\addplot+ table [x=distance, y=qualityRhombicDodecaLower]{results/distanceQuality.csv};}
%        \addplot+ table [x=distance, y=qualityAverageEquiTiled]{results/distanceQuality.csv};


        \coordinate(top)at(rel axis cs:0,1);
    \end{axis}
    \node[anchor=south] at (top.north) {\phantom{F}};%trick to avoid jumping of the graph
\end{tikzpicture}

    \caption{A caption}
    \label{fig:dist_quality}
\end{figure}

\begin{figure}
    \pgfplotscreateplotcyclelist{My color list}{%
    {color1,solid, very thick},%
    {color2,densely dashed, very thick},%
    {color3,densely dotted, very thick},%
    {color4,dash pattern=on 4pt off 1pt on 4pt off 4pt, very thick}%
}

\pgfplotsset{every axis legend/.append style={
        at={(0,0.97)},
        anchor=south west,
        draw=none,
        fill=none,
        legend columns=4,
        column sep=5pt,
        /tikz/every odd column/.append style={column sep=0cm},
        font=\scriptsize
}}
\pgfplotsset{grid style={dashed,gray}}
\pgfplotsset{minor grid style={gray!20}}
\pgfplotsset{major grid style={dotted}}

\tikzsetnextfilename{distance_quality_psnr}
\begin{tikzpicture}
    \begin{axis}[
            ylabel={PSNR gap (in dB)},
            xlabel={Orthodromic distance (in distance units)},
            width=1.05\linewidth,
            height=0.5\linewidth,
            cycle list name={My color list},
            legend cell align=left,
            xmin = 0,
            xmax = 3.14,
%            ymajorgrids,
        ]
        \pgfplotsextra{\begin{scope}[on layer=axis background]
                \draw[draw=gray!20] (axis cs:0,0) -- (axis cs:3.14,0);
                \node[rounded corners, fill=gray!20, font=\tiny, inner sep=2pt, anchor=west] at (axis cs:0,0) {\textit{\vphantom{lj}uniEqui}};
            \end{scope}
        }

        \addplot+ table [x=distance,
        y=qualityEquirectangularTiledLower]{results/distanceQuality_psnr.csv};
        \addplot+ table [x=distance,
        y=qualityCubeMapLower]{results/distanceQuality_psnr.csv};
        \addplot+ table [x=distance,
        y=qualityPyramidLower]{results/distanceQuality_psnr.csv};
        \addplot+ table [x=distance,
        y=qualityRhombicDodecaLower]{results/distanceQuality_psnr.csv};
        \legend{Equirec,CubeMap,Pyramid,Dodeca}

    \end{axis}
\end{tikzpicture}

    \caption{A caption}
    \label{fig:dist_quality_psnr}
\end{figure}


\section{Results}

\begin{figure}
    \usetikzlibrary{pgfplots.statistics}
\newcommand\storelabel[2]{\expandafter\xdef\csname label#1\endcsname{#2}}
\newcommand\getlabel[1]{\csname label#1\endcsname}

\newcommand{\newBoxplot}[5]{%
    \addplot+[ boxplot prepared={%
            lower whisker={#1}, lower quartile={#2}, median={#3},
            upper quartile={#4}, upper whisker={#5},% average={#6},
%            box extend=0.5,  % height of box
%            whisker extend=0.5, % height of whiskers
%            every box/.style={thin,draw=black},
%            every whisker/.style={black,thick},
%            every median/.style={black,thick},
%            every average/.style={draw=red, /tikz/mark=* },
    /pgf/number format/precision=2 } ]
    coordinates {}
    %node[left,black] at
    %(boxplot box cs: \boxplotvalue{lower whisker},0.5)
    %{\tiny\pgfmathprintnumber{\boxplotvalue{lower whisker}}}
    %node[right,black] at
    %(boxplot box cs: \boxplotvalue{upper whisker},0.5)
    %{\tiny\pgfmathprintnumber{\boxplotvalue{upper whisker}}}
    ;
} %end of \newBoxplot definition

\newcommand{\newBoxPlotFromCdfFile}[2]{%
%    \pgfplotstabletypeset[ignore chars={\#}]{#1}
    \pgfplotstablegetelem{0}{#2}\of{#1}
    \edef\min{\pgfplotsretval}
    \storelabel{min#1#2}{\min}
    \pgfplotstablegetelem{50}{#2}\of{#1}
    \edef\qmin{\pgfplotsretval}
    \storelabel{qmin#1#2}{\qmin}
    \pgfplotstablegetelem{100}{#2}\of{#1}
    \edef\med{\pgfplotsretval}
    \storelabel{med#1#2}{\med}
    \pgfplotstablegetelem{150}{#2}\of{#1}
    \edef\qmax{\pgfplotsretval}
    \storelabel{qmax#1#2}{\qmax}
    \pgfplotstablegetelem{200}{#2}\of{#1}
    \edef\max{\pgfplotsretval}
    \storelabel{max#1#2}{\max}
    %\newBoxplot{\pgfmathparse{\getlabel{min#1#2}/\MaxNbFrame}\pgfmathresult}{\pgfmathparse{\getlabel{qmin#1#2}/\MaxNbFrame}\pgfmathresult}{\pgfmathparse{\getlabel{med#1#2}/\MaxNbFrame}\pgfmathresult}{\pgfmathparse{\getlabel{qmax#1#2}/\MaxNbFrame}\pgfmathresult}{\pgfmathparse{\getlabel{max#1#2}/\MaxNbFrame}\pgfmathresult}
    \newBoxplot{\getlabel{min#1#2}}{\getlabel{qmin#1#2}}{\getlabel{med#1#2}}{\getlabel{qmax#1#2}}{\getlabel{max#1#2}}
}


\pgfplotscreateplotcyclelist{My color list}{%
    {black!50,solid,thick},%
    {black!85,solid,thick}%
}

\def\ymin{0.7}

\begin{tikzpicture}
    \begin{axis}[
            boxplot/draw direction=y,
            ylabel={\acs{MS-SSIM}},
            width=1.05\linewidth,
            height=0.5\linewidth,
            boxplot={
                draw position={1/3 + floor(0.01+\plotnumofactualtype/2) + 1/3*mod(\plotnumofactualtype,2)},
                box extend=0.30,
            },
            %x=2cm,
            xtick={0,1,2,...,10},
            x tick label as interval,
            xticklabels={%
                {{\tiny Good~Bad}},%
                {{\tiny Good~Bad}},%
                {{\tiny Good~Bad}},%
                {{\tiny Good~Bad}},%
            },
            x tick label style={
                text width=2.5cm,
                align=center
            },
            cycle list name={My color list},
            legend cell align=left,
            xmin = 0,
            xmax = 4,
            ymin = \ymin,
            ymax = 1,
            axis on top,
        ]
        \pgfplotsextra{\begin{scope}[on layer=axis background]
                \draw[fill=gray!14,draw=gray!14] (axis cs:1,\ymin) rectangle
                (axis cs:2,1);
                \draw[fill=gray!14,draw=gray!14] (axis cs:3,\ymin) rectangle
                (axis cs:4,1);
            \end{scope}
        }
        \newBoxPlotFromCdfFile{results/cdfQuality.csv}{goodEquirectangularTiled}
        \newBoxPlotFromCdfFile{results/cdfQuality.csv}{badEquirectangularTiled}

        \newBoxPlotFromCdfFile{results/cdfQuality.csv}{goodCubMap}
        \newBoxPlotFromCdfFile{results/cdfQuality.csv}{badCubMap}

        \newBoxPlotFromCdfFile{results/cdfQuality.csv}{goodPyramidal}
        \newBoxPlotFromCdfFile{results/cdfQuality.csv}{badPyramidal}

        \newBoxPlotFromCdfFile{results/cdfQuality.csv}{goodRhombicDodeca}
        \newBoxPlotFromCdfFile{results/cdfQuality.csv}{badRhombicDodeca}

    \end{axis}
    \begin{axis}[
            width=1.05\linewidth,
            height=0.5\linewidth,
            xmin=0,xmax=4,
            ymin=\ymin,ymax = 1,
            axis x line*=top,
            axis y line=none,
            enlargelimits=false,
            hide y axis,
            xtick={0,1,2,...,10},
            x tick label as interval,
            xticklabels={%
                { EquiTiled },%
                { Cube },%
                { Pyramid },%
                { Dodeca },%
            },
            x tick label style={
                text width=2.5cm,
                align=center
            },
        ]
    \end{axis}

\end{tikzpicture}

    \caption{A caption}
    \label{fig:box_plot}
\end{figure}

\begin{figure}
    \tikzsetnextfilename{distance_quality}
\begin{tikzpicture}
   \pgfplotscreateplotcyclelist{My color list}{%
       {color1,solid, very thick},%
       {color2, dashed, very thick},%
       {color3, densely dashed, very thick},%
       {color4, densely dotted, very thick},%
   }

   \pgfplotsset{every axis legend/.append style={
           at={(-0.03,0.97)},
   anchor=south west,
   draw=none,
   fill=none,
   legend columns=4,
   column sep=6pt,
   %font = \scriptsize,
   /tikz/every odd column/.append style={column sep=0cm},
   %font=\tiny
   }}
   \pgfplotsset{grid style={dashed,gray}}
   \pgfplotsset{minor grid style={dotted,red!20!gray}}
   \pgfplotsset{major grid style={dotted,green!50!black}}

    \begin{axis}[
            ylabel={MS-SSIM},
            xlabel={Orthodromic distance},
            width=0.95\linewidth,
            height=0.5\linewidth,
            cycle list name={My color list},
            legend cell align=left,
            xmin = 0,
            xmax = 3.14,
            ymin = 0.88,
            ymax = 1,
            ymajorgrids,
            yminorgrids,
        ] 
         \pgfplotsextra{\begin{scope}[on layer=axis background]
                \draw[draw=color5] (axis cs:0,0.955777) -- (axis cs:3.14,0.955777);
                \node[rounded corners, fill=color5,
                			font=\tiny, inner sep=2pt,
                			anchor=west, text=white] at (axis cs:0,0.955777) {\textit{\vphantom{lj}uniEqui}};
            \end{scope}
            \begin{scope}[on layer=axis background]
                   \draw[draw=black, dashed] (axis cs:1.57,0) -- (axis cs:1.57,1);
                   \node[anchor=north west,xshift=2pt,yshift=-2pt,inner sep=2pt,legendinfigure] at (axis cs:1.57, 1) {\footnotesize{During a 2\,s segment, 86\,\% of the users never explore this zone}};
               \end{scope}
        }

        \only<.(1)->{\legend{Equirec,CubeMap,Pyramid,Dodeca}}
        \only<+->{\addplot+ table [x=distance, y=qualityEquirectangularTiledLower]{results/distanceQuality.csv};}
        \only<+->{\addplot+ table [x=distance, y=qualityCubeMapLower]{results/distanceQuality.csv};}
        \only<+->{\addplot+ table [x=distance, y=qualityPyramidLower]{results/distanceQuality.csv};}
        \only<+->{\addplot+ table [x=distance, y=qualityRhombicDodecaLower]{results/distanceQuality.csv};}
%        \addplot+ table [x=distance, y=qualityAverageEquiTiled]{results/distanceQuality.csv};


        \coordinate(top)at(rel axis cs:0,1);
    \end{axis}
    \node[anchor=south] at (top.north) {\phantom{F}};%trick to avoid jumping of the graph
\end{tikzpicture}

    \caption{A caption}
    \label{fig:dist_quality}
\end{figure}

\begin{figure}
    \pgfplotscreateplotcyclelist{My color list}{%
    {color1,solid, very thick},%
    {color2,densely dashed, very thick},%
    {color3,densely dotted, very thick},%
    {color4,dash pattern=on 4pt off 1pt on 4pt off 4pt, very thick}%
}

\pgfplotsset{every axis legend/.append style={
        at={(0,0.97)},
        anchor=south west,
        draw=none,
        fill=none,
        legend columns=4,
        column sep=5pt,
        /tikz/every odd column/.append style={column sep=0cm},
        font=\scriptsize
}}
\pgfplotsset{grid style={dashed,gray}}
\pgfplotsset{minor grid style={gray!20}}
\pgfplotsset{major grid style={dotted}}

\tikzsetnextfilename{distance_quality_psnr}
\begin{tikzpicture}
    \begin{axis}[
            ylabel={PSNR gap (in dB)},
            xlabel={Orthodromic distance (in distance units)},
            width=1.05\linewidth,
            height=0.5\linewidth,
            cycle list name={My color list},
            legend cell align=left,
            xmin = 0,
            xmax = 3.14,
%            ymajorgrids,
        ]
        \pgfplotsextra{\begin{scope}[on layer=axis background]
                \draw[draw=gray!20] (axis cs:0,0) -- (axis cs:3.14,0);
                \node[rounded corners, fill=gray!20, font=\tiny, inner sep=2pt, anchor=west] at (axis cs:0,0) {\textit{\vphantom{lj}uniEqui}};
            \end{scope}
        }

        \addplot+ table [x=distance,
        y=qualityEquirectangularTiledLower]{results/distanceQuality_psnr.csv};
        \addplot+ table [x=distance,
        y=qualityCubeMapLower]{results/distanceQuality_psnr.csv};
        \addplot+ table [x=distance,
        y=qualityPyramidLower]{results/distanceQuality_psnr.csv};
        \addplot+ table [x=distance,
        y=qualityRhombicDodecaLower]{results/distanceQuality_psnr.csv};
        \legend{Equirec,CubeMap,Pyramid,Dodeca}

    \end{axis}
\end{tikzpicture}

    \caption{A caption}
    \label{fig:dist_quality_psnr}
\end{figure}


\rowcolors{2}{gray!25}{white}
\begin{table}[t]
\scriptsize
\centering
\begin{tabular}{r@{\hspace{2pt}}|c|c@{\hspace{5pt}}c|c@{\hspace{5pt}}c|c@{\hspace{5pt}}c}
% \rowcolor{gray!50}
 \multicolumn{2}{c|}{~} & \multicolumn{2}{c|}{high quality} & \multicolumn{2}{c|}{mid quality} & \multicolumn{2}{c}{low quality} \\
%\rowcolor{gray!50}
 & \# tiles & \# tiles & res. & \# tiles & res. & \# tiles & res. \\ \hline \hline
 \textbf{equi.} & 64 & 9 & 1 & 40 & 0.5 & 15 & 0.25 \\
 \textbf{cube} & 6 & 1 & 1 & 4 & [0.4,0.7] & 1 & 0.25\\
 \textbf{pyramid} & 5 & 1 & 1 & 4 & --- & 0 & ---\\
 \textbf{dodeca.} & 12 & 2 & 1 & 8 & [0.4--0.7] & 2 & 0.25 
\end{tabular}
\caption{tile quality arrangement}\label{tab:quality}
\end{table}


\section{Discussion and Conclusion}

We can use \AD{focal}{} statistics and saliency maps to select better \acp{QEC}.
% eye attention

We have to extend the tests for a large set of videos and for many other parameter settings: adjusting quality by spatially or CRF ? how many versions (QECs)? toward a rate-distortion figure?

We have to see what are the options if we have a lot of bandwidth (case of terahertz antennas).



\newpage
%%%%%%%%%%%%%%%%%%%%%%%%%%%%%%%
%\bibliographystyle{IEEEtran}
%\bibliographystyle{IEEEbib}
\bibliographystyle{abbrvnat}  
%\bibliographystyle{abbrv}  
\bibliography{biblio}

\end{document}

